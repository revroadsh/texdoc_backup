\documentclass[cn,blue]{elegantbook}
\input {d:/tex/preamble}

\excludecomment{solution}
\excludecomment{answer}

\begin{document}

\chapter{因式分解(20200416)}%
\label{cha:因式分解}
\section{知识要点}%
\label{sec:知识要点}
\begin{enumerate}
    \item 定义:
        \begin{enumerate}
            \item 和差化积:将整式的和差形式化为乘积形式.
            \item 因式分解与整式乘法互为逆运算. \(x(x+y) \leftrightharpoons x^2+xy \),从左到右为整式乘除,从右到左为因式分解.
        \end{enumerate}
    \item 因式分解的方法
        \begin{enumerate}
            \item 提公因式法:(1)先提系数的最大公约数; (2)再提各字母最低的次数; (3)提取后观察是否可进一步分解
            \item 公式法:(1)完全平方公式; (2)平方差公式
            \item 分组分解法
            \item 十字相乘法
                \begin{enumerate}
                    \item 只适用于如\(ax^2+bx+c\)形式的二次三项式
                    \item 原理:\((x+a)(x+b)=x^2+(a+b)x+ab, (ax+b)(cx+d)=acx^2+(ad+bc)x+bd\),
                    \item 两种情况:(1)二次项系数为1; (2)二次项系数不为1
                    \item 为一元二次方程,一元二次函数打基础
                \end{enumerate}
        \end{enumerate}
\end{enumerate}
\section{例题}%
\label{sec:例题}
\subsection{提公因式法}%
\begin{problem}
    \(2x^3y^2 -4x^2y +8x^2y^2\)
\end{problem}
\begin{solution}
    答案:\(2x^3y^2 -4x^2y +8x^2y^2=2x^2y(xy-2+4y)\)\\
\end{solution}

\subsection{分组分解法}%
\label{ssub:分组分解法}
\subsubsection{分组后能直接提公因式}%
\label{par:分组后能直接提公因式}
\begin{problem}
    \begin{enumerate}
        \item \(am+an+bm+bn\)
        \item \(a^2-ab+ac-bc\)
        \item \(xy-x-y+1\)
    \end{enumerate}
\end{problem}
\begin{solution}
\begin{enumerate}
    \item \(am+an+bm+bn=(a+b)(m+n)\)
    \item \(a^2-ab+ac-bc=(a-b)(a+c)\)
    \item \(xy-x-y+1=(x-1)(y-1)\)
\end{enumerate}
\end{solution}

\subsubsection{分组后能直接用公式}%
\label{par:分组后能直接用公式}
\begin{problem}
    \begin{enumerate}
        \item \(x^2 -y^2+ax+ay\)
        \item \(a^2 -c^2-2ab+b^2\)
        \item \(x^2-x-9y^2-3y\)
        \item \(ax^2 -bx^2+bx-ax+a-b\)
        \item \(a^2 -6ab+12b+9b^2-4a\)
        \item \(4a^2x-4a^2y-b^2x+b^2y\)
        \item 在有理数或实数范围内分解:\(x^4-4\)
        \item \(a^2 -2a+b^2-2b+2ab+1\)
        \item \((a+c)(a-c)+b(b-2a)\)
    \end{enumerate}
\end{problem}
\begin{solution}
    \begin{enumerate}
        \item \(x^2 -y^2+ax+ay=(x+y)(x-y)+a(x+y)=(x+y)(x-y+a)\)
        \item \(a^2 -c^2-2ab+b^2=(a^2-2ab+b^2-c^2)=(a-b)^2-c^2=(a-b+c)(a-b-c)\)
        \item \(x^2-x-9y^2-3y=(x^2-9y^2)-(x+3y)=(x+3y)(x-3y)-(x+3y)=(x+3y)(x-3y-1)\)
        \item \(ax^2 -bx^2+bx-ax+a-b=x^2(a-b)-x(a-b)+(a-b)=(a-b)(x^2-x+1)\)
        \item \(a^2 -6ab+12b+9b^2-4a=(a^2-6ab+9b^2)+(12b-4a)=(a-3b)^2+4(3b-a)=(3b-a)^2+4(3b-a)=(3b-a)(3b-a+4)\)
        \item \(4a^2x-4a^2y-b^2x+b^2y=4a^2(x-y)-b^2(x-y)=(x-y)(4a^2-b^2)=(x-y)(2a+b)(2a-b)\)
        \item 在有理数或实数范围内分解:\(x^4-4\)
            \begin{enumerate}
                \item 有理数范围:\(x^4-4=(x^2+2)(x^2-2)\)
                \item 实数范围:\(x^4-4=(x^2+2)(x^2-2)=(x^2+2)(x+\sqrt{2})(x-\sqrt{2})\)
            \end{enumerate}
        \item \(a^2 -2a+b^2-2b+2ab+1=(a^2+2ab+b^2)-(2a+2b)+1=(a+b)^2-2(a+b)+1=(a+b-1)^2\).注意:将\((a+b)\)看成一个整体,运用完全平方公式
        \item \((a+c)(a-c)+b(b-2a)=a^2-c^2+b^2-2ab=(a^2-2ab+b^2)-c^2=(a-b)^2-c^2=(a-b+c)(a-b-c)\).注意:原式有乘除,但不是因式分解,需要先打开再重新组合.
    \end{enumerate}
\end{solution}
\subsection{十字相乘法}%
\label{sub:十字相乘法}
\subsubsection{二次项系数为1}%
\label{ssub:二次项系数为1}
原理:\(x^2 +(a+b)x+ab=(x+a)(x+b)\).即两个数的和是一次项系数,两个数的积是常数项
\begin{problem}
    \begin{enumerate}
        \item \(x^2 -5x+4\)
        \item \(x^2 +5x+6\)
        \item \(x^2 -7x+6\)
        \item \(x^2 +14x+24\)
        \item \(a^2 -15a+36\)
        \item \(x^2 +4x-5\)
        \item \(x^2 +x-2\)
        \item \(y^2 -2y-15\)
        \item \(x^2 -10x-24\)
    \end{enumerate}
\end{problem}

\begin{solution}
    \begin{enumerate}
        \item \(x^2 -5x+4=(x-1)(x-4)\)
        \item \(x^2 +5x+6=(x+2)(x+3)\)
        \item \(x^2 -7x+6=(x-1)(x-6)\)
        \item \(x^2 +14x+24=(x+2)(x+12)\)
        \item \(a^2 -15a+36=(a-12)(a-3)\)
        \item \(x^2 +4x-5=(x-1)(x+5)\)
        \item \(x^2 +x-2=(x-1)(x+2)\)
        \item \(y^2 -2y-15=(y-5)(y+3)\),注意变量为\(y\)
        \item \(x^2 -10x-24=(x-12)(x+2)\)
    \end{enumerate}
\end{solution}
\subsubsection{二次项系数不为1}%
原理:\(ax^2 +bx+c=(a_1x_0+c_1)(a_2x_0+c_2)\),其中\(a_1\times a_2=a, c_1\times c_2=c,\text{且}a_1\times c_2+a_2\times c_1=b\)
\begin{problem}
    \begin{enumerate}
        \item \(3x^2 -11x+10\)
        \item \(5x^2 +7x-6\)
        \item \(3x^2 -7x+2\)
        \item \(10x^2 -17x+3\)
        \item -\(6y^2 +11y+10\)
        \item \(2x^2 -7xy+6y^2\)
        \item \(x^2 y^2-3xy+2\)
        \item \(15x^2 +7xy-4y\)
        \item \(12x^2 -11xy-15y^2\)
    \end{enumerate}
\end{problem}

\begin{solution}
    \begin{enumerate}
        \item \(3x^2 -11x+10=(x-2)(3x-5)\)
        \item \(5x^2 +7x-6=(5x-3)(x+2)\)
        \item \(3x^2 -7x+2=(x-2)(3x-1)\)
        \item \(10x^2 -17x+3=(2x-3)(5x-1)\)
        \item -\(6y^2 +11y+10=-(6y^2-11y-10)=-(2y-5)(3y+2)\).注意先提负号.
        \item \(2x^2 -7xy+6y^2=(x-2y)(2x-3y)\)
        \item \(x^2 y^2-3xy+2=(xy-1)(xy-2)\)
        \item \(15x^2 +7xy-4y=(3x-y)(5x+4y)\)
        \item \(12x^2 -11xy-15y^2=(3x-5y)(4x+3y)\)
    \end{enumerate}
\end{solution}

\subsection{综合}%
\label{sub:综合}
\begin{problem}
    \begin{enumerate}
        \item \((x+y)^2-3(x+y)-10\)
        \item \((a+b)^2-4a-4b+3\)
        \item \(m^2-4mn+4n^2-3m+6n+2\)
        \item \(x^2 +4xy+4y^2-2x-4y-3\)
        \item \(4x^2-4xy-6x+3y+y^2-10\)
    \end{enumerate}
\end{problem}

\begin{solution}
    \begin{enumerate}
        \item \((x+y)^2-3(x+y)-10=(x+y-5)(x+y+2)\)
        \item \((a+b)^2-4a-4b+3=(a+b)^2-4(a+b)+3=(a+b-1)(a+b-3)\)
        \item \(m^2-4mn+4n^2-3m+6n+2=(m-2n)^2-3(m-2n)+2=(m-2n-1)(m-2n-2)\)
        \item \(x^2 +4xy+4y^2-2x-4y-3=(x+2y)^2-2(x+2y)-3=(x+2y-3)(x+2y+1)\)
        \item \(4x^2-4xy-6x+3y+y^2-10=(4x^2-4xy+y^2)+(3y-6x)-10=(2x-y)^2-3(2x-y)-10=(2x-y-5)(2x-y+3)\)
    \end{enumerate}
\end{solution}

\section{习题}%
\label{sec:习题}

\begin{enumerate}
    \item 下列变形是因式分解的是(\qquad)\\
        \begin{tasks}(2) 
            \task \((3 - x)(3 + x) = 9 - x^2\)
            \task \(m^3 - mn^2 = m(m + n)(m - n)\)
            \task \(( y +1)( y - 3) = -(3 - y)( y +1)\)
            \task \(4yz - 2y^2 z + z = 2y(2z - yz) + z\)
        \end{tasks}
\begin{solution}
    答案:B.\\
    解析:A为整式乘法,C原式已经是因式分解完成,D右边是未完全展开的整式\\
\end{solution}
    \item 下列多项式中能用平方差公式分解因式的是(\qquad)\\
    \begin{tasks}(4) 
    \task \(a^2 + (-b)^2\)
    \task \(5m^2 - 20mn\)
    \task \(-x^2 - y^2\)
    \task \(-x^2 + 9\)
    \end{tasks}
\begin{solution}
    答案:d.\\
    解析:ac都是平方和.d是\(3^2-x^2\)\\
\end{solution}
    \item 若\(( p - q)^2 - (q - p)^3 = (q - p)^2 \cdot  E\) ,则E 是(\qquad)\\
    \begin{tasks}(4) 
    \task \(1- q - p\)
    \task \(q - p\)
    \task \(1+ p - q\)
    \task \(1+ q - p\)
    \end{tasks}
\begin{solution}
    答案:C\\
    解析:\(( p - q)^2 - (q - p)^3 =(q-p)^2-(q-p)^3=(q-p)^2\cdot (1-q+p)=(q-p)^2\cdot (1+p-q)= (q - p)^2 \cdot  E,\therefore E=1+p-q\)\\
\end{solution}
              
    \item 一个多项式分解因式的结果是\((b^3 + 2)(2 - b^3 )\),那么这个多项式是(\qquad)\\
    \begin{tasks}(4) 
    \task \(b^6 - 4\)
    \task \(4 - b^6\)
    \task \(b^6 + 4\)
    \task \(-b^6 - 4\)
    \end{tasks}
\begin{solution}
    答案:B.\\
    解析:平方差公式,同底数幂的乘法\\
\end{solution}
              
    \item 把多项式\(m^2 (a - 2) + m(2 - a)\)分解因式等于(\qquad)\\
    \begin{tasks}(2) 
    \task \((a - 2)(m^2 + m)\)
    \task \((a - 2)(m^2 - m)\)
    \task \(m(a - 2)(m -1)\)
    \task \(m(a - 2)(m +1)\)
    \end{tasks}

\begin{solution}
    答案:C\\
    解析:\(m^2 (a - 2) + m(2 - a)=m^2(a-2)-m(a-2)=(a-2)(m^2-m)=m(a-2)(m-1)\)\\
\end{solution}
          
    \item 下列多项式中,含有因式\(( y +1)\)的多项式是(\qquad)\\
    \begin{tasks}(4) 
    \task \(y^2 - 2xy - 3x^2\)
    \task \(( y +1)^2 - ( y -1)^2\)
    \task \(( y +1)^2 - ( y^2 -1)\)
    \task \(( y +1)^2 + 2( y +1) +1\)
    \end{tasks}
\begin{solution}
    答案:C\\
    解析:\(A=(y-3x)(y+x);B=4y\text{直接打开或利用平方差公式};C=y^2+2y+1-y^2-1=2y+2=2(y+1); D=(y+1+1)^2=(y+2)^2\)\\
\end{solution}
          
    \item 已知多项式\(2x^2 + bx + c\)分解因式为\(2(x - 3)(x +1)\),则\(b,c\)的值为(\qquad)\\
    \begin{tasks}(2) 
    \task \(b = 3,c = -1\)
    \task \(b = -6,c = 2\)
    \task \(b = -6,c = -4\)
    \task \(b = -4,c = -6\)
    \end{tasks}
\begin{solution}
    答案:D.\\
    解析:\(2(x - 3)(x +1)=2(x^2-2x-3)=2x^2-4x-6\)\\
\end{solution}
          
    \item 若将\(x^2 + px + q\)分解因式为\((x - 3)(x + 5)\),则\(p\) 为(\qquad)\\
    \begin{tasks}(4) 
    \task -15
    \task -2
    \task 8
    \task 2
    \end{tasks}

\begin{solution}
    答案:D\\
    解析:\(p=-3+5=2,q=-3\times 5=-15\)\\
\end{solution}

    \item  \(\triangle ABC \)的三边\(a,b,c\) 满足\(a^2 - 2bc = c^2 - 2ab\),则 \(\triangle ABC \) 是(\qquad)\\
    \begin{tasks}(4) 
    \task 等腰三角形
    \task 直角三角形
    \task 等边三角形
    \task 锐角三角形
    \end{tasks}

\begin{solution}
    答案:A\\
    解析:\(a^2 - 2bc = c^2 - 2ab \Rightarrow a^2-2bc-c^2+2ab=a^2-c^2+(2ab-2bc)=(a+c)(a-c)+2b(a-c)=(a-c)(a+c+2b)=0.\)\\
    \(\because a,b,c\text{为边长,都大于0},\therefore a+c+2b>0,\therefore a-c=0 \Rightarrow a=c,\therefore \)是等腰三角形.
\end{solution}

    \item 已知\(a = 2002x + 2006,b = 2002x + 2007,c = 2002x + 2008\),则多项式\(a^2 + b^2 + c^2 - ab - bc - ca\)的值为(\qquad)\\
    \begin{tasks}(4) 
    \task 0
    \task 1
    \task 2
    \task 3
    \end{tasks}

\begin{solution}
    答案:D\\
    解析:\(a^2 +b^2+c^2-ab-bc-ca=\frac{1}{2}(2a^2+2b^2+2c^2-2ab-2bc-2ca)=\frac{1}{2}(a^2-2ab+b^2+b^2-2bc+c^2+a^2-2ac+c^2)=\frac{1}{2}[(a-b)^2+(b-c)^2+(a-c)^2]=\frac{1}{2}(1^2+1^2+2^2)=3\). 常用技巧之一.\\
\end{solution}

    \item 已知:\(ab \ne 0,a^2 + ab - 2b^2 = 0\),那么 \(\frac{2a-b}{2a+b}\) 的值为\tkt{\qquad}.
\begin{solution}
    答案:\(\frac{1}{3}\text{或}\frac{5}{3}\)\\
    解析: \(ab \ne 0 \Rightarrow a\ne 0,b\ne 0\)\\
    \((a^2+ab-2b^2=0 \Rightarrow (a+2b)(a-b)=0 \because a\ne 0,b\ne 0,\therefore \Rightarrow a=b\text{或}a=-2b\)\\
    代入\(\frac{2a-b}{2a+b}\),分类讨论,求得值.
    注意分类讨论.
\end{solution}
    \item 分解因式:\(x(a - b)^{2n} + y(b - a)^{2n+1} =\)\tkt{\qquad}.
\begin{solution}
    答案:\((b-2a)^{2n}(x+by-ay)\)\\
    解析:利用偶次方底数正负号无关的性质.
\end{solution}
    \item 观察右图,根据图形的面积关系,不需要连其他的线,便可以得到一个用来分解因式的公式,这个公式是.
        \begin{figure}[H]
        \begin{center}
            \includegraphics[width=0.2\textwidth]{pic/13.png}
        \end{center}
        \end{figure}
\begin{solution}
    答案:\(a^2 +b^2+2ab=(a+b)^2\)\\
    解析:完全平方公式的几何意义.注意必须从和的形式写成积的形式才符合体重所说的分解因式.\\
\end{solution}
        
    \item 若\(x^2 + 2(m - 3)x +16\)是完全平方公式,则\(m=\tkt{\qquad}\).
\begin{solution}
    答案:-1或7\\
    解析:注意分类讨论,完全平方公式有\((a\pm b)^2\)两种形式.即一次项的系数\(2(m-3)=\pm 2ab=\pm 2\times 1\times 4=8 \Rightarrow m-3=\pm 4 \Rightarrow m=-1\text{或}m=7\)\\
\end{solution}
    \item 若\((x^2 + y^2 )(x^2 + y^2 -1) =12\),则\(x^2 + y^2 =\tkt{\qquad}.\)
\begin{solution}
    答案:4\\
    解析:\(\text{令}x^2+y^2=a,(x^2 +y^2)(x^2+y^2-1)=12\Rightarrow a(a-1)=12 \Rightarrow a^2-a-12=0 \Rightarrow a=4\text{或}a=-3. \because a=x^2+y^2\ge 0,\therefore -3\text{舍去},x^2+y^2=4\).整体代换时,必须注意代换式的取值范围.\\
\end{solution}
    \item 已知\(a,b,c,d\) 为非负整数,且\(ac + bd + ad + bc =1997\),则\(a + b + c + d = \tkt{\qquad}\).
\begin{solution}
    答案:1998\\
    解析:原式\(=a(c+d)+b(c+d)=(a+b)(c+d)=1997,\because 1997\text{是质数},\therefore a+b,c+d\text{只能是}1,1997,\therefore a+b+c+d=1+1997=1998\)\\
\end{solution}
\end{enumerate}

\section{中考真题}%

\begin{problem}
    \begin{enumerate}
        \item 江西2008 分解因式:\(x^3-4x=\)
        \item 江西2010 分解因式:\(2a^2-8=\)
        \item 江西2011 分解因式:\(x^3-x=\)
        \item 江西2013 分解因式:\(x^2-4=\)
        \item 江西2016 分解因式:\(ax^2-ay^2=\)
        \item 把多项式\(x^2+ax+b\)分解因式,得\((x+1)(x-3)\)则\(a,b\)的值分别是(\qquad)\\
            \begin{tasks}(4) 
                \task \(a=2,b=3\)
                \task \(a=-2,b=-3\)
                \task \(a=-2,b=3\)
                \task \(a=2,b=-3\)
            \end{tasks}
        \item 下列运算错误的是 (\qquad)\\
            \begin{tasks}(4) 
                \task \(a+2a=3a\)
                \task \((a^2)^3=a^6\)
                \task \(a^2 \times a^3=a^5\)
                \task \(a^6 \div a^3=a^2\)
            \end{tasks}
        \item 下列计算正确的是(\qquad)\\
            \begin{tasks}(4) 
                \task \(3a+4b=7ab\)
                \task \((ab^3)^3=ab^6\)
                \task \((a+2)^2=a^2+4\)
                \task \(x^{12} \div x^6=x^6\)
            \end{tasks}
        \item 当\(1<a<2\)时,代数式\(|a-2|+|1-a|\)的值是(\qquad)\\
            \begin{tasks}(4) 
                \task -1
                \task 1
                \task 3
                \task -3
            \end{tasks}
        \item 下列计算正确的是(\qquad)\\
            \begin{tasks}(4) 
            \task \(x^2 \times x^3=x^5\)
                \task \(x^6+x^6=x^{12}\)
                \task \((x^2)^3=x^5\)
                \task \(x^{-1}=x\)
            \end{tasks}
        \item 分解因式:\(16-x^2=\)(\qquad)\\
            \begin{tasks}(4) 
                \task \((4-x)(4+x)\)
                \task \((x-4)(x+4)\)
                \task \((8+x)(8-x)\)
                \task \((4-x)^2\)
            \end{tasks}
        \item 将下列多项式因式分解,结果中不含有因式\(a+1\)的是(\qquad)\\
            \begin{tasks}(4) 
                \task \(a^2-1\)
                \task \(a^2+a\)
                \task \(a^2+a-2\)
                \task \((a+2)^2-2(a+2)+1\)
            \end{tasks}
        \item 分解因式:\(a^3-9a=\)
        \item 分解因式:\(a^3-16a=\)
        \item 因式分解:\(a^2-6a+9=\)
        \item 分解因式:\(x^2-36=\)
        \item 把多项式\(9a^3-ab^2\)分解因式的结果是
        \item 把多项式\(ax^2+2a^2x+a^3\)分解因式的结果是
        \item 分解因式:\(a^2-9=\)
        \item 分解因式\(3m^4-48=\)
        \item 分解因式:\(xy^2-x=\)
        \item 分解因式:\(ab^4-4ab^3+4ab^2=\)
        \item 分解因式:\(ax^2-ay^2=\)
        \item 分解因式:\(2a^2+4a+2=\)
        \item 分解因式:\((m+1)(m-9)+8m=\)
    \end{enumerate}
\end{problem}

\begin{solution}
    \begin{enumerate}
        \item 江西2008 分解因式:\(x^3-4x=x(x^2-4)=x(x+2)(x-2)\)
        \item 江西2010 分解因式:\(2a^2-8=2(a^2-4)=2(a+2)(a-2)\)
        \item 江西2011 分解因式:\(x^3-x=x(x^2-1)=x(x+1)(x-1)\)
        \item 江西2013 分解因式:\(x^2-4=(x+2)(x-2)\)
        \item 江西2016 分解因式:\(ax^2-ay^2=a(x^2-y^2)=a(x+y)(x-y)\)
        \item 把多项式\(x^2+ax+b\)分解因式,得\((x+1)(x-3)\)则\(a,b\)的值分别是(B)\\
            \(a=-3+1=2,b=-3\times 1=-3\)
        \item 下列运算错误的是 (D)\\
            \begin{tasks}(4) 
                \task \(a+2a=3a\)
                \task \((a^2)^3=a^6\)
                \task \(a^2 \times a^3=a^5\)
                \task \(a^6 \div a^3=a^2\)
            \end{tasks}
            解析:\(a^6\div a^3=a^3\)
        \item 下列计算正确的是(D)\\
            \begin{tasks}(4) 
                \task \(3a+4b=7ab\)
                \task \((ab^3)^3=ab^6\)
                \task \((a+2)^2=a^2+4\)
                \task \(x^{12} \div x^6=x^6\)
            \end{tasks}
            解析:\(A=7a, B=a^3b^6; C=a^2+4a+4\)
        \item 当\(1<a<2\)时,代数式\(|a-2|+|1-a|\)的值是(C)\\
            \begin{tasks}(4) 
                \task -1
                \task 1
                \task 3
                \task -3
            \end{tasks}
            解析:原式=\(2-a+a-1=3\)
        \item 下列计算正确的是(A)\\
            \begin{tasks}(4) 
            \task \(x^2 \times x^3=x^5\)
                \task \(x^6+x^6=x^{12}\)
                \task \((x^2)^3=x^5\)
                \task \(x^{-1}=x\)
            \end{tasks}
            解析:\(B=2x^6, C=x^6, D=\frac{1}{x}\)
        \item 分解因式:\(16-x^2=\)(A)\\
            \begin{tasks}(4) 
                \task \((4-x)(4+x)\)
                \task \((x-4)(x+4)\)
                \task \((8+x)(8-x)\)
                \task \((4-x)^2\)
            \end{tasks}
        \item 将下列多项式因式分解,结果中不含有因式\(a+1\)的是(C)\\
            \begin{tasks}(4) 
                \task \(a^2-1\)
                \task \(a^2+a\)
                \task \(a^2+a-2\)
                \task \((a+2)^2-2(a+2)+1\)
            \end{tasks}
            解析:\(A=(a+1)(a-1), B=a(a+1), C=(a+2)(a-1), D=(a+2-1)^2=(a+1)^2\)
        \item 分解因式:\(a^3-9a=a(a^2-9)=a(a+3)(a-3)\)
        \item 分解因式:\(a^3-16a=a(a^2-16)a(a+4)(a-4)\)
        \item 因式分解:\(a^2-6a+9=(a-3)^2\)
        \item 分解因式:\(x^2-36=(x+6)(x-6)\)
        \item 把多项式\(9a^3-ab^2\)分解因式的结果是\(=a(9a^2-b^2)=a(3a+b)(3a-b)\)
        \item 把多项式\(ax^2+2a^2x+a^3\)分解因式的结果是\(=a(x^2+2ax+a^2)=a(x+a)^2\)
        \item 分解因式:\(a^2-9=(a+3)(a-3)\)
        \item 分解因式\(3m^4-48=3(m^4-16)=3(m^2+4)(m^2-4)=3(m^2+4)(m+2)(m-2)\)
        \item 分解因式:\(xy^2-x=x(y^2-1)=x(y+1)(y-1)\)
        \item 分解因式:\(ab^4-4ab^3+4ab^2=ab^2(b^2-4b+4)=ab^2(b-2)^2\)
        \item 分解因式:\(ax^2-ay^2=a(x^2-y^2)=a(x+y)(x-y)\)
        \item 分解因式:\(2a^2+4a+2=2(a^2+2a+1)=a(a+1)^2\)
        \item 分解因式:\((m+1)(m-9)+8m=m^2-8m-9+8m=m^2-9=(m+3)(m-3)\)
    \end{enumerate}
\end{solution}
\end{document}
