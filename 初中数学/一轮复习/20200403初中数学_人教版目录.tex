\documentclass[cn,blue,10pt]{elegantbook}
\input{d:/tex/preamble}

\begin{document}

\maketitle

\tableofcontents

\mainmatter

\part{七年级上}
\chapter{有理数(算术)}
\section{正数和负数}
\section{有理数}
\section{有理数的乘除法}
\section{有理数的乘方}
\chapter{整式的加减(代数)}
\section{整式}
\section{整式的加减}
\chapter{一元一次方程(方程)}
\section{从算式到方程}
\section{解一元一次方程:合并同类项与移项}
\section{解一元一次方程:去括号与去分母}
\chapter{几何图形初步(几何)}
\section{几何图形}
\section{直线、射线、线段}
\section{角}
\part{七下}
\chapter{相交线与平行线(几何)}
\section{相交线}
\section{平行线及其判定}
\section{平行线的性质}
\section{平移}
\chapter{实数(算术)}
\section{平方根}
\section{立方根}
\section{实数}
\chapter{平面直角坐标系(解析几何)}
\section{平面直角坐标系}
\section{坐标方法的简单应用}
\chapter{二元一次方程组(方程)}
\section{二元一次方程组}
\section{消元--解二元一次方程组}
\section{实际问题与二元一次方程组}
\section{*三元一次方程组的解法}
\chapter{不等式与不等式组(方程)}
\section{不等式}
\section{一元一次不等式}
\section{一元一次不等式组}
\chapter{数据的收集、整理与描述(统计与概率)}
\section{统计调查}
\section{直方图}
\section{课题学习:从数据谈节水}
\part{八上}
\chapter{三角形(几何)}
\section{与三角形有关的线段}
\section{与三角形有关的角}
\section{多边形及其内角和}
\chapter{全等三角形(几何)}
\section{全等三角形}
\section{三角形全等的判定}
\section{角的平分线的性质}
\section{轴对称(几何)}
\chapter{轴对称}
\section{画轴对称图形}
\section{等腰三角形}
\section{课题学习:最短路径问题}
\chapter{整式的乘法与因式分解(代数)}
\section{整式的乘法}
\section{乘法公式}
\section{因式分解}
\chapter{分式(代数)}
\section{分式}
\section{分式的运算}
\section{分式方程}
\part{八下}
\chapter{二次根式(代数)}
\section{二次根式}
\section{二次根式的乘除}
\section{二次根式的加减}
\chapter{勾股定理(几何)}
\section{勾股定理}
\section{勾股定理的逆定理}
\section{平行四边形(几何)}
\chapter{平行四边形}
\section{特殊的平行四边形}
\chapter{一次函数(函数)}
\section{函数}
\section{一次函数}
\section{课题学习:选择方案}
\chapter{数据的分析(统计与概率)}
\section{数据的集中趋势}
\section{数据的波动程度}
\section{课题学习:体质健康测试中的数据分析}
\part{九上}
\chapter{一元二次方程(方程)}
\section{一元二次方程}
\section{解一元二次方程}
\section{实际问题与一元二次方程}
\include {math_junior_22}
\chapter{旋转(几何)}
\section{图形的旋转}
\section{中心对称}
\section{课题学习:图案设计}
\chapter{圆(几何)}
\section{圆的有关性质}
\section{点和圆、直线和圆的位置关系}
\section{正多边形和圆}
\section{弧长和扇形的面积}
\chapter{概率初步(统计与概率)}
\section{随机事件与概率}
\section{用列举法求概率}
\section{用频率估计概率}
\part{九下}
\chapter{反比例函数(函数)}
\section{反比例函数}
\section{实际问题与反比例函数}
\chapter{相似(几何)}
\section{图形的相似}
\section{相似三角形}
\section{位似}
\chapter{锐角三角函数(函数)}
\section{锐角三角函数}
\section{解直角三角形及其应用}
\chapter{投影与视图(几何)}
\section{投影}

\subsection{知识要点}%
\label{sub:知识要点}

\subsubsection{光源}%
\label{ssub:光源}

\(\begin{cases} \text{灯:点光源}\\ \text{太阳光:平行光源} \\  \end{cases}\)

\subsubsection{投影}%
\label{ssub:投影}
\begin{enumerate}
    \item 定义:\(\begin{cases} \text{点光源:中心投影}\\ \text{平行光源:平行投影}\\  \end{cases}\)
    \item 性质:\(\begin{cases} \text{中心投影:物体垂直地面时,越长越近}\\ \text{平行光源:平行投影}\\  \end{cases}\)
\end{enumerate}
\section{三视图}
\section{课题学习:制作立体模型}

\part{综合}

\chapter{几何模型}%
\label{cha:几何模型}

\section{相似模型}%
\label{sec:相似模型}

\subsection{A字形和8字形}%
\label{sub:a_8模型}

\begin{figure}[H]
    \centering
    \includegraphics[height=6cm]{fig/1001001.png}
    \caption{A字形和8字形相似}%
\end{figure}
条件: \(\angle 1 = \angle 2\)

结论: \(\triangle ADE \sim \triangle ABC\)

应用:通过平行,得出两大类,四小类相似.

\subsection{共边共角型}%
\label{sub:共边共角型}
\begin{figure}[H]
    \centering
    \includegraphics[height=4cm]{fig/1001002.png}
    \caption{共边共角型相似}%
    \label{fig:相似A8模型}
\end{figure}
条件: \(\angle 1 = \angle 2\) 

结论: \(\triangle ACD \sim \triangle ABC\) 

其他结论: \(AC^2 = AD \cdot AB\)

\subsection{一线三等角型}%
\label{sub:一线三等角型}
\begin{figure}[H]
    \centering
    \includegraphics[height=3cm]{fig/1001003.png}
    \caption{一线三等角}%
    \label{fig:相似A8模型}
\end{figure}

条件: \(\angle 1 = \angle 2\) 

结论: \(\triangle ACD \sim \triangle ABC\) 

证明:\\
    \(\because \angle ACE + \angle DCE = \angle B + \angle A,\text{ 且 } \angle B = \angle ACE\) \\
    \(\therefore \angle DCE = \angle A \) \\
    \(\therefore \triangle ABC \sim \triangle CDE\) \\
    图2,图3同理可得

分析: 一线三等角模型中,难点在于当已知三个相等角时,容易忽略隐含的其他相等的角.图2中的三垂直相似模型应用较多,看见该模型时,应能立刻看出其中的相似三角形.

\subsection{倒数型}%
\label{sub:倒数型}
\begin{figure}[h]
    \centering
    \includegraphics[height=3cm]{fig/1001004.png}
    \caption{倒数型相似}%
    \label{fig:倒数型相似}
\end{figure}
条件: \(AF // DE // BC\) 

结论: \(\dfrac{1}{AF}+\dfrac{1}{BC}=\dfrac{1}{DE}\) 

证明:\\
\(\because  AF // DE // BC\) \\
\(\therefore \triangle BDE \sim \triangle BAF, \triangle ADE \sim \triangle ABC\)\\
\(\therefore \dfrac{DE}{AF}=\dfrac{BD}{AB},\dfrac{DE}{BC}=\dfrac{AD}{AB}.\)\\
\(\therefore \dfrac{DE}{AF}+\dfrac{DE}{BC}=\dfrac{BD}{AB}+\dfrac{AD}{AB}=\dfrac{AB}{AB}=1.\)\\
即\(\dfrac{DE}{AF}+\dfrac{DE}{BC}=1,\)\\
\(\therefore \dfrac{1}{AF}+\dfrac{1}{BC}=\dfrac{1}{DE}\),(两边同除DE).

分析: 倒数型模型由两个A型相似模型相加而得.

\subsection{与圆有关的简单相似}%
\label{sub:与圆有关的简单相似}
\begin{figure}[h]
    \centering
    \includegraphics[height=4cm]{fig/1001004.png}
    \caption{倒数型相似}%
    \label{fig:倒数型相似}
\end{figure}

结论:

图1,由同弧所对圆周角相等,得\(\triangle PAC \sim \triangle PDB\)

图2,由圆的内接四边形的一个外角等于它的内对角,得\(\triangle ABD \sim \triangle AEC.\)

图3,已知AB切\(\odot O\)于点A,则\(\triangle BAD \sim \triangle BCA\)

证明:
\begin{figure}[H]
    \centering
    \includegraphics[height=4cm]{fig/1001007.png}
    \label{fig:1001007}
\end{figure}
图3:如图,过A作直径AE,连接DE,\\
则有\(\angle EAD+\angle E=90^\circ \) \\
又\(\angle BAD + \angle EAD = 90^\circ,\)\\
\(\therefore \angle BAD=\angle E=\angle C.\)\\
\(\therefore \triangle BAD \sim \triangle BCA\)

圆的内接四边形的性质
\begin{figure}[H]
    \centering
    \includegraphics[height=4cm]{fig/1001006.png}
    \caption{圆的内接四边形}%
    \label{fig:fig/}
\end{figure}
以圆内接四边形ABCD为例,圆心为O,延长AB至E,AC、BD交于P,则
\begin{enumerate}
    \item 圆内接四边形的对角互补:∠BAD+∠DCB=180°,∠ABC+∠ADC=180°
    \item 圆内接四边形的任意一个外角等于它的内对角:∠CBE=∠ADC
    \item 圆心角的度数等于所对弧的圆周角的度数的两倍:∠AOB=2∠ACB=2∠ADB
    \item 同弧所对的圆周角相等:∠ABD=∠ACD
    \item 圆内接四边形对应三角形相似:△ABP∽△DCP(三个内角对应相等)
    \item 相交弦定理:AP×CP=BP×DP
    \item 托勒密定理:AB×CD+AD×CB=AC×BD
\end{enumerate}

\subsection{相似与旋转}%
\label{sub:相似与旋转}

\begin{figure}[h]
    \centering
    \includegraphics[height=3cm]{fig/1001011.png}
    \caption{旋转型相似}%
    \label{fig:倒数型相似}
\end{figure}
条件: 如图1,已知\(DE // BC,\triangle ADE\)绕点A旋转一定角度,连接BD,CE,得到图2.

结论: \(\triangle ABD \sim \triangle ACE\) 

证明:\\
\(\because  DE // BC\) \\
\(\therefore \dfrac{AD}{AB}=\dfrac{AE}{AC},\) \\
图2中,\(\angle DAE = \angle BAC,\therefore \angle BAD=\angle CAE\) \\
\(\therefore \triangle ABD \sim \triangle ACE\)

分析: 本模型难度较高,常出现在压轴题中,以直角三角形为背景出题,综合性较强.考察知识点有相似,旋转,勾股定理,三角函数等.

附:三角形相似的判定
\begin{enumerate}
    \item 两角对应相等,两三角形相似.
    \item 两边对应成比例且夹角相等,两个三角形相似. 
    \item 三边对应成比例,两个三角形相似.
\end{enumerate}

\section{辅助圆}%
\label{sec:辅助圆}
\subsection{共端点,等线段模型}%
\label{sub:共端点_等线段模型}

\begin{figure}[h]
    \centering
    \includegraphics[height=3cm]{fig/1001013.png}
    \caption{共端点,等线段构造辅助圆}%
    \label{fig:倒数型相似}
\end{figure}
条件: 三条线段共端点且长度相等,即\(OA=OB=OC\) 

结论: 三条线段的端点在同一个圆上.
证明:\\

分析: 可以构造辅助圆,利用圆的性质快速解决角度问题.如圆周角等于圆心角的一半\(\angle ACB = \dfrac{1}{2}\angle AOB, \angle BAC = \dfrac{1}{2} \angle BOC\) 


\subsection{直角三角形共斜边模型}%
\label{sub:直角三角形共斜边模型}

\begin{figure}[h]
    \centering
    \includegraphics[height=6cm]{fig/1001022.png}
    \caption{直角三角形共斜边模型}%
    \label{fig:倒数型相似}
\end{figure}
条件: 两个直角三角形共斜边(同侧或异侧) 

结论: A,B,C,D四点共圆. 

证明:\\
取AB中点O,根据直角三角形斜边中线等于斜边一半,可得\(OA=OB=OC=OD\),所以四点共圆.

分析:证明四点共圆之后,可以利用圆的性质证明角度等量关系,是证明角度相等的重要途径之一. 



\section{8字模型与飞镖模型}%
\label{sec:8字模型与飞镖模型}

\subsection{角的8字模型}%
\label{sub:角的8字模型}

\begin{figure}[h]
    \centering
    \includegraphics[height=3cm]{fig/1001026.png}
    \caption{角的8字模型}%
    \label{fig:倒数型相似}
\end{figure}
条件: O是AC和BD的交点 

结论: \(\angle A + \angle D = \angle B + \angle C\) 

证明:\\
对顶角相等,三角形内角和等于180度.

分析:8字模型常用来在几何综合题中推导角度. 
\subsection{角的飞镖模型}%
\label{sub:角的飞镖模型}

\begin{figure}[h]
    \centering
    \includegraphics[height=3cm]{fig/1001030.png}
    \caption{角的飞镖模型}%
    \label{fig:倒数型相似}
\end{figure}
条件: 如图 

结论: \(\angle D=\angle A+\angle B+\angle C\) 

证明:\\
\includegraphics[height=3cm]{fig/1001031.png}

连接BC,三角形内角和等于180度.\\
\(\angle 2+\angle 4+\angle D=180^\circ\)\\
\(\angle 1+\angle 2+\angle 3+\angle 4+\angle A=180^\circ\)\\
\(\therefore \angle A+\angle 1+\angle 3=\angle D\)

分析:常在几何综合题中用于推导角度.
\subsection{边的8字模型}%
\label{sub:边的8字模型}

\begin{figure}[h]
    \centering
    \includegraphics[height=3cm]{fig/1001028.png}
    \caption{边的8字模型}%
    \label{fig:倒数型相似}
\end{figure}
条件: O是AC和BD的交点 

结论: \(AC+BD>AD+BC\) 

证明:\\
\(\because OA+OD>AD, OB+OC>BC\)\\
\(\therefore OA+OD+OB+OC>BC+AD\)
即:\(AC+BD>AD+BC\)

分析: 
\subsection{边的飞镖模型}%
\label{sub:边的飞镖模型}

\begin{figure}[h]
    \centering
    \includegraphics[height=3cm]{fig/1001032.png}
    \includegraphics[height=3cm]{fig/1001033.png}
    \caption{倒数型相似}%
    \label{fig:倒数型相似}
\end{figure}
条件:如图 

结论: \(AB+AC>BD+CD\) 

证明:\\
三角形两边之和大于第三边.\\
如图,延长BD交AC于点E.\\
\(\because CE+DE>CD\)\\
\(\therefore BD+CE+DE>BD+CD\),即\(BE+CE>BD+CD\)\\
又\(AB+AE>BE\)\\
\(\therefore AB+AE+CE>BD+CD\),即\(AB+AC>BD+CD\)\\

分析:用来证明边的不等式 


\section{角平分线四大模型}%
\label{sec:角平分线四大模型}

\begin{figure}[h]
    \centering
    \includegraphics[height=3cm]{fig/1001004.png}
    \caption{倒数型相似}%
    \label{fig:倒数型相似}
\end{figure}
条件: \(\) 

结论: \( \) 

证明:\\

分析: 
\subsection{角平分线上的点向两边做垂线}%
\label{sub:角平分线上的点向两边做垂线}

\begin{figure}[h]
    \centering
    \includegraphics[height=3cm]{fig/1001004.png}
    \caption{倒数型相似}%
    \label{fig:倒数型相似}
\end{figure}
条件: \(\) 

结论: \( \) 

证明:\\

分析: 

\subsection{截取构造对称全等}%
\label{sub:截取构造对称全等}

\begin{figure}[h]
    \centering
    \includegraphics[height=3cm]{fig/1001004.png}
    \caption{倒数型相似}%
    \label{fig:倒数型相似}
\end{figure}
条件: \(\) 

结论: \( \) 

证明:\\

分析: 
\subsection{角平分线+垂直构造等腰三角形}%
\label{sub:角平分线_垂直构造等腰三角形}

\begin{figure}[h]
    \centering
    \includegraphics[height=3cm]{fig/1001004.png}
    \caption{倒数型相似}%
    \label{fig:倒数型相似}
\end{figure}
条件: \(\) 

结论: \( \) 

证明:\\

分析: 
\subsection{角平分线+平行线}%
\label{sub:角平分线_平行线}

\begin{figure}[h]
    \centering
    \includegraphics[height=3cm]{fig/1001004.png}
    \caption{倒数型相似}%
    \label{fig:倒数型相似}
\end{figure}
条件: \(\) 

结论: \( \) 

证明:\\

分析: 





\section{截长补短}%
\label{sec:截长补短}

\begin{figure}[h]
    \centering
    \includegraphics[height=6cm]{fig/1001046.png}
    \caption{截长补短法}%
    \label{fig:倒数型相似}
\end{figure}
条件: 已知线段AB,CD,EF,求三者之间的数量关系 

结论:

通过截长(图2),在EF上截取EG=AB;证明EG和AB的关系.

或补短(图3),延长AB至H点,使BH=CD,证明AH和EF的关系.

证明:\\

分析: 截长补短法适用于求证线段的和差倍分关系.截长是指在长线段中截取一段等于已知线段;补短是指将短线的延长,延长部分等于已知线段.常用于等腰三角形,角平分线等情况,采用截长补短法构造全等三角形来完成证明.

\section{手拉手模型}%
\label{sec:手拉手模型}

\begin{figure}[h]
    \centering
    \includegraphics[height=3cm]{fig/1001004.png}
    \caption{倒数型相似}%
    \label{fig:倒数型相似}
\end{figure}
条件: \(\) 

结论: \( \) 

证明:\\

分析: 
\section{三垂直}%
\label{sec:三垂直}

\begin{figure}[h]
    \centering
    \includegraphics[height=3cm]{fig/1001004.png}
    \caption{倒数型相似}%
    \label{fig:倒数型相似}
\end{figure}
条件: \(\) 

结论: \( \) 

证明:\\

分析: 
\section{将军饮马}%
\label{sec:将军饮马}

\begin{figure}[H]
    \centering
    \includegraphics[height=6cm]{fig/1001036.png}
    \includegraphics[height=6cm]{fig/1001038.png}
    \includegraphics[height=3cm]{fig/1001040.png}
    \caption{两定点与直线上的动点间的距离问题}%
    \label{fig:倒数型相似}
\end{figure}
条件: \(\) 

结论: \( \) 

证明:\\

分析: 
\subsection{定直线与两定点}%
\label{sub:定直线与两定点}

\begin{figure}[h]
    \centering
    \includegraphics[height=3cm]{fig/1001004.png}
    \caption{倒数型相似}%
    \label{fig:倒数型相似}
\end{figure}
条件: \(\) 

结论: \( \) 

证明:\\

分析: 
\subsection{角与定点}%
\label{sub:角与定点}

\begin{figure}[h]
    \centering
    \includegraphics[height=3cm]{fig/1001050.png}\\
    \includegraphics[height=3cm]{fig/1001052.png}\\
    \includegraphics[height=3cm]{fig/1001054.png}
    \caption{角与定点}%
    \label{fig:倒数型相似}
\end{figure}
条件: \(\) 

结论: \( \) 

证明:\\

分析: 
\subsection{两定点一定长}%
\label{sub:两定点一定长}

\begin{figure}[h]
    \centering
    \includegraphics[height=3cm]{fig/1001056.png}\\
    \includegraphics[height=3cm]{fig/1001058.png}
    \caption{倒数型相似}%
    \label{fig:倒数型相似}
\end{figure}
条件: \(\) 

结论: \( \) 

证明:\\

分析: 

\subsection{胡不归模型}%
\label{sub:胡不归模型}

背景:

一个身在他乡的小伙子,得知父亲病危的消息后便日夜赶路回家。然而,当他气喘吁吁地来到父亲的面前时,老人刚刚咽气了。人们告诉他,在弥留之际,老人在不断喃喃地叨念:“胡不归?胡不归?”

A是出发地,B是目的地;AC是一条驿道,而驿道靠目的地的一侧是沙地。为了急切回家,小伙子选择了直线路程AB。但是,他忽略了在驿道上行走要比在砂土地带行走快的这一因素。如果他能选择一条合适的路线(尽管这条路线长一些,但是速度可以加快),是可以提前抵达家门的
\begin{figure}[H]
    \centering
    \includegraphics[height=4cm]{fig/1001070.png}
    \caption{胡不归问题背景}%
    \label{fig:倒数型相似}
\end{figure}

解析:
构造某个角的正弦值等于系数k(k<1时)或\(\dfrac{1}{k}\)(k>1时)
\begin{figure}[H]
    \centering
    \includegraphics[height=4cm]{fig/1001072.png}
    \caption{胡不归问题背景}%
    \label{fig:倒数型相似}
\end{figure}

例题:


分析: 
\begin{enumerate}
    \item 费马与笛卡尔讨论光的折射现象时,发现胡不归问题与光的折射现象一致,即光总是走时间最短路线.
    \item \(PA+k\cdot PB\)型最值问题,k=1时转化为PA+PB之和最短,就是将军饮马模型;当k为\(\ne 1\)的正数时,如果P在直线上运动,就是胡不归问题;若P在圆上运动,就是``阿氏圆''问题.
\end{enumerate}


\subsection{阿氏圆问题}%
\label{sub:阿氏圆问题}

背景:

“阿氏圆”又称“阿波罗尼斯圆”,已知平面上两点A、B,则所有满足PA=k·PB(k≠1)的点P的轨迹是一个圆,这个轨迹最早由古希腊数学家阿波罗尼斯发现,故称“阿氏圆”。

如图1,\(\odot O\)的半径为r,点A,B都在\(\odot O\)外,P为圆上一动点,已知\(r=k\cdot OB\),连接PA,PB,当\(``PA+k\cdot PB''\)的值最小时,P点的位置如何确定?
\begin{figure}[H]
    \centering
    \includegraphics[height=4cm]{fig/1001074.png}
    \includegraphics[height=4cm]{fig/1001076.png}
    \caption{阿氏圆问题}%
    \label{fig:阿氏圆问题}
\end{figure}

解析:
在线段OB上截取OC,使OC=\(k\cdot r\),则\(\triangle BPO\sim \triangle PCO\),即\(k\cdot PB=PC\).\(PA+k\cdot PB\)最小值转化为\(PA+PC\)的最小值,其中A与C为定点,P为动点,当A,P,C三点共线时,\(PA+PC\)最小.

例题:


分析: 
\begin{enumerate}
    \item \(PA+k\cdot PB\)型最值问题,k=1时转化为PA+PB之和最短,就是将军饮马模型;当k为\(\ne 1\)的正数时,如果P在直线上运动,就是胡不归问题;若P在圆上运动,就是``阿氏圆''问题.
\end{enumerate}


\section{半角模型}%
\label{sec:半角模型}

\begin{figure}[h]
    \centering
    \includegraphics[height=6cm]{fig/1001024.png}
    \caption{角含半角模型}%
    \label{fig:倒数型相似}
\end{figure}
条件: \(\angle 2=\dfrac{1}{2}\angle AOB;\angle 4=\angle 3;OA=OB; OF'=OF\) 

结论: 旋转全等: \(\triangle OAF'\cong\triangle OBF\);对称全等:\(\triangle OEF \cong \triangle OEF'\) 

证明:\\
\(\because OF'=OF, \angle 3=\angle 4, OA=OB\)\\
\(\therefore \triangle OAF'\cong \triangle OBF(SAS)\)\\
\(\because \angle 2 = \dfrac{1}{2}\angle AOB\)\\
\(\therefore \angle 1 + \angle 3 = \angle 2\),又\(\angle 4 = \angle 3\)\\
\(\therefore \angle 1 + \angle 3 = \angle 2\),即\(\angle F'OE = \angle EOF\)\\
由\(OF'=OF, OE=OE, \angle F'OE = \angle EOF\)\\
可得\(\triangle OEF \cong \triangle OEF' (SAS)\)

分析:
\begin{enumerate}
    \item 半角模型:存在两个角,其中一个是另一个的一半,且这两个角共顶点.
    \item \(\triangle OAF'\)由\(\triangle OBF\)绕O点旋转\(\angle AOB\)而来.B点旋转至A点,F点旋转至F'点.
    \item 在遇到角含半角时,首先构造旋转全等,再寻找对称全等.构造旋转全等的方法是从O点作射线OF',使\(\angle F'OA = \angle FOB\),在射线上截取\(OF'=OF\).
    \item 通过旋转全等和轴对称全等,一般用来证明线段和差关系.
    \item 常见的半角模型是90度含45度,120度含60度.
\end{enumerate}

\section{蚂蚁行程:立体图形展开的最短路径}%
\label{sec:蚂蚁行程}

\begin{figure}[h]
    \centering
    \includegraphics[height=3cm]{fig/1001016.png}
    \caption{立体图形展开的最短路径}%
    \label{fig:倒数型相似}
\end{figure}
条件: 蚂蚁从点A沿圆柱爬行一周,到点B的最短路径.

结论: 最短路径就是展开图中AB'的长,\(AB'=\sqrt{ AA'^2 + A'B'^2 }\) 

证明:\\

分析: 立体图形表面最短路径问题的关键是正确展开立体图形,然后利用``两点之间线段最短''或``两边之和大于第三边''准确找出最短路径.

\section{中点四大模型}%
\label{sec:中点四大模型}

\subsection{倍长中线或类中线构造全等}%
\label{sub:倍长中线或类中线构造全等}

\begin{figure}[h]
    \centering
    \includegraphics[height=3cm]{fig/1001004.png}
    \caption{倒数型相似}%
    \label{fig:倒数型相似}
\end{figure}
条件: \(\) 

结论: \( \) 

证明:\\

分析: 
\subsection{等腰底边三线合一}%
\label{sub:等腰底边三线合一}

\begin{figure}[h]
    \centering
    \includegraphics[height=3cm]{fig/1001004.png}
    \caption{倒数型相似}%
    \label{fig:倒数型相似}
\end{figure}
条件: \(\) 

结论: \( \) 

证明:\\

分析: 
\subsection{直角三角形直角边中点}%
\label{sub:直角三角形直角边中点}

\begin{figure}[H]
    \centering
    \includegraphics[height=3cm]{fig/1001020.png}
    \caption{直角三角形斜边中线}%
    \label{fig:倒数型相似}
\end{figure}
条件: \(\triangle \) 

结论: \( \) 

证明:\\

分析: 
\subsection{直角三角形斜边中点}%
\label{sub:直角三角形斜边中点}

\begin{figure}[h]
    \centering
    \includegraphics[height=3cm]{fig/1001020.png}
    \caption{直角三角形斜边中线}%
    \label{fig:倒数型相似}
\end{figure}
条件: \(\triangle ABC\)是\(Rt\triangle\),D是斜边中点 

结论: \(CD=\dfrac{1}{2}AB\);\(\triangle ACD\)和\(\triangle BCD\)是等腰三角形

证明:\\
构造外接圆,AB为直径,CD为半径.

分析: 直角三角形中,遇见斜边中点,一般可作斜边中线,利用斜边中线等于斜边一半,证明线段间的数量关系.反过来,如果中线等于对应边的一半,则该三角形为直角三角形.该模型经常与中位线定理综合应用.




\section{圆中的辅助线}%
\label{sec:圆中的辅助线}

\subsection{构造等腰三角形}%
\label{sub:构造等腰三角形}

\begin{figure}[h]
    \centering
    \includegraphics[height=3cm]{fig/1001004.png}
    \caption{倒数型相似}%
    \label{fig:倒数型相似}
\end{figure}
条件: \(\) 

结论: \( \) 

证明:\\

分析: 

\subsection{构造直角三角形}%
\label{sub:构造直角三角形}

\begin{figure}[h]
    \centering
    \includegraphics[height=3cm]{fig/1001004.png}
    \caption{倒数型相似}%
    \label{fig:倒数型相似}
\end{figure}
条件: \(\) 

结论: \( \) 

证明:\\

分析: 
\subsection{切线相关辅助线}%
\label{sub:切线相关辅助线}

\begin{figure}[h]
    \centering
    \includegraphics[height=3cm]{fig/1001004.png}
    \caption{倒数型相似}%
    \label{fig:倒数型相似}
\end{figure}
条件: \(\) 

结论: \( \) 

证明:\\

分析: 




\end{document}
