\documentclass[cn,blue,12pt]{elegantbook}
\input {d:/tex/preamble}

\excludecomment{note}
\excludecomment{solution}
\renewcommand \tkt[1]{{\CJKunderline[hidden=true, skip=true, thickness=1pt]{#1}}}

\begin{document}

\chapter{分式方程及其应用}%
\label{cha:分式方程及其应用}
\begin{note}
    人教:八上第十五章P 149-P 156; 北师: 八下第五章 P 125 - P 130
\end{note}
\section{知识要点}%
\label{sec:知识要点}

\begin{zsyd}
\item 分式方程的概念及解法
    \begin{zsyd}
    \item 分式方程的概念:\tkt{分母中含有未知数}的方程.
    \item 解分式方程
        \begin{zsyd}
        \item 基本思想:分式方程\(\xlongrightarrow{\text{化简}}\)整式方程
        \item 一般步骤:
            \begin{zsyd}
            \item 化为\tkt{整式方程:去分母}
            \item 解整式方程
            \item \tkt{检验}:\(\begin{cases} \text{\tkt{最简公分母为0}}:x=a\text{不是分式方程的解}\\ \text{\tkt{最简公分母不为0}}:x=a\text{是分式方程的解} \\  \end{cases}\)
            \end{zsyd}
        \item 技巧:去分母时, 先确定\tkt{最简公分母}; 若分母是多项式, 要进行\tkt{因式分解}; 若分子是多项式,则需将其\tkt{看作一个整体},\tkt{加括号}后再进行下一步运算.
        \item 分式方程的增根:分式方程的增根是\tkt{去分母后的整式方程解得根,但此根却是使分式方程的分母为0的根}.
        \item 分式方程无解:可能是\tkt{解为增根}, 也可能是\tkt{去分母后的整式方程无解}.
        \end{zsyd}
    \end{zsyd}
\item 分式方程的实际应用
    \begin{zsyd}
    \item 解题步骤:实际问题\(\xlongrightarrow[\mbox{\text{设未知数}}]{\mbox{\text{找等量关系}}}\)列分式方程 \(\to\)解分式方程 \(\to\)双检验 \(\to\)作答.
    \item 【易错警示】 解分式方程必须``双检验'' :  \ding{172}检验\tkt{是否是分式方程的解}; \ding{173}检验\tkt{是否符合实际意义}.
    \item 常见问题的关系式
        \begin{zsyd}
        \item 行程问题\tkt{\(\begin{cases} \dfrac{\text{路程}}{\text{速度}}= \text{时间}\\ \dfrac{\text{同一路程}}{\text{慢速}} - \dfrac{\text{同一路程}}{\text{快速}} = \text{时间差} \end{cases}\)}
        \item \(\dfrac{\text{工作总量}}{\text{\tkt{工作效率}}}=\text{工作时间}\)
        \item 购买分配类问题:\tkt{\(\dfrac{\text{总价}}{\text{单价}}=\text{数量}\)}
        \end{zsyd}
    \end{zsyd}
\end{zsyd}
\section{重难点突破}%
\label{sec:重难点突破}
\textbf{分式方程的解法}:
\begin{liti}[resume]
\item 解方程: \(\frac{1}{x-5}=\frac{x+1}{5-x}+2\)\\
解: 方程两边同时乘以\((x-5)\),\\
得: \(1=-(x+1)+2\) .................................. 第一步\\
去括号得: \(1=-x-1+2\),..................................第二步\\
移项得: \(x=-1+2-1\),..................................... 第二步\\
解得: \(x=0\), ......................................................第四步\\
检验: 当\(x=0\)时,\(x-5 \ne 0\), .......................... 第五步\\
\(\therefore \)原分式方程的根为\(x=0\) ........................... 第六步\\
【查找错因】上述解答过程是从第\tkt{一}步开始出现错误的, 错误原因是\tkt{忘给2乘以(x-5)}.\\
【自主解答】
\vspace{40 pt}\\
【易错警示】给方程两边同乘以最简公分母时, 不要给常数项或整式部分漏乘.\\
\begin{solution}
    \(x=12\)
\end{solution}
\end{liti}
\section{中考真题}%
\label{sec:中考真题}
\subsubsection{命题点:分式方程的解法}(仅\(2010\)年考)%
\begin{zhenti}[resume]
\item (2018江西18题7分)解方程:\(\frac{x-2}{x+2}+\frac{4}{x^2-4}=1.\)
\begin{solution}
        \(x=3\)
\end{solution}
\item (2019玉林)解方程:\(\frac{x}{x-1}-\frac{3}{(x-1)(x+2)}=1.\)
\begin{solution}
        \(x=1\)是增根,方程无解.
\end{solution}
\end{zhenti}

\subsubsection{命题点分式方程的实际应用}(仅2019年考)%

\begin{zhenti}[resume]
\item (\(2019\)江西\(11\)题\(3\)分)斑马线前\(``\)车让人\('' \),不仅体现着一座城市对生命的尊重,也直接反映着城市的文明程度.如图,某路口的斑马线路段\(A-B-C\)横穿双向行驶车道,其中\(AB=BC = 6\)米,在绿灯亮时,小明共用11秒通过\(AC\),其中通过\(BC\)的速度是通过\(AB\)速度的2倍,求小明通过\(AB\)时的速度.设小明通过\(AB\)时的速度是\(x\)米/秒,根据题意列方程得:\tkt{\(\frac{6}{x}+\frac{6}{1.2x}=11\)}  .
\item (\(2019\)广州)甲、乙二人做某种机械零件,已知每小时甲比乙少做\(8\)个,甲做\(120\)个所用的时间与乙做\(150\) 个所用的时间相等,设甲每小时做\(x\)个零件,下列方 程正确的是(\qquad)\\
\begin{tasks}(2)
\task \(\frac{120}{x}=\frac{150}{x-8}\)
\task \(\frac{120}{x+8}=\frac{150}{x}\)
\task \(\frac{120}{x-8}=\frac{150}{x}\)
\task \(\frac{120}{x}=\frac{150}{x+8}\)
\end{tasks}
\begin{solution}
    D
\end{solution}
\item 班级组织同学乘大巴车前往``研学旅行''基地开展爱国教育活动,基地离学校有\(90\)公里,队伍8:00从学校出发.苏老师因有事情,8:30从学校自驾小车以大巴\(1.5\)倍的速度追赶,追上大巴后继续前行,结果比队伍提前\(15\)分钟到达基地.问:\\
(1)大巴与小车的平均速度各是多少?\\
(2)苏老师追上大巴的地点到基地的路程有多远?
\begin{solution}
    (1)\(\frac{90}{x}-\frac{90}{1.5x}=\frac{45}{60} \Rightarrow x=40\),大巴40\kilo\metre\per\hour ,小汽车60 \kilo\metre\per\hour .\\
    (2) \(\frac{x}{40}-\frac{x}{60}=0.5 \Rightarrow x=60\),到基地路程\(=90-60=30\)公里.
\end{solution}
\end{zhenti}

\section{练习}%
\label{sec:练习}

\subsubsection{回归教材}%
\begin{shiti}
\item 李庄村原来用\( 10 hm^2 \)耕地种植粮食作物,用\( 80 hm^2 \)种植经济作物.为了增加粮食作物的种植面积,该村计划将部分种植经济作物的耕地改为种植粮食作物,使得粮食作物的种植面积与经济作物的种植面积之比为\( 5:7. \)设有\( x hm^2 \)种植经济作物的耕地改为种植粮食作物,那么\( x \)满足怎样的分式方程?
\begin{solution}
        \(\frac{10+x}{80-x}=\frac{5}{7}\)
\end{solution}
\item 有两块面积相同的小麦试验田,第一块使用原品种,第二块使用新品种,分别收获小麦\( 12 000 kg \)和\( 14 000 kg\),已知第一块试验田每公顷的产量比第二块少\( 1 500 kg. \)如果设第一块试验田每公顷的产量为\( x \kilogram \),那么\( x \)满足怎样的分式方程?
\begin{solution}
\(\frac{12000}{x}=\frac{14000}{x+1500}\)
\end{solution}
\item 某运输公司需要装运一批货物,由于机械设备没有及时到位,只好先用人工装运,\( 6 h\)完成了一半任务;后来机械装运和人工装运同时进行,\( 1 h \)完成了后一半任务.如果设单独采用机械装运\( x h \)可以完成后一半任务,那么\( x \)满足怎样的分式方程?
\begin{solution}
        \(\frac{1}{6}+\frac{1}{x}=1\)或\(\frac{1}{12}+\frac{1}{2x}=\frac{1}{2}\)
\end{solution}
\item 从甲地到乙地有两条公路:一条是全长\( 600 km \)的普通公路,另一条是全长\( 480 km \)的高速公路.某客车在高速公路上行驶的平均速度比在普通公路上快\( 45 km/h\),由高速公路从甲地到乙地所需的时间是由普通公路从甲地到乙地所需时间的一半.如果设该客车由高速公路从甲地到乙地所需的时间为\( x h\),那么\( x \)满足怎样的分式方程?
\begin{solution}
        \(\frac{480}{x}-\frac{600}{2x}=45\)
\end{solution}
\item 某市为治理污水,需要铺设一段全长为\( 3 000 m \)的污水排放管道.为了尽量减少施工对城市交通所造成的影响,实际施工时每天的工效比原计划增加\( 25\%\),结果提前\( 30 \)天完成这一任务.实际每天铺设多长管道?
\begin{solution}
        25m.\\
        原计划每天x, \(\frac{3000}{x}-\frac{3000}{(1+25\%)x=30} \Rightarrow \)
\end{solution}
\item 某质检部门抽取甲、乙两厂相同数量的产品进行质量检测,结果甲厂有\( 48 \)件合格产品,乙厂有\( 45 \)件合格产品,甲厂的合格率比乙厂高\( 5\%\),求甲厂的合格率.
\begin{solution}
        甲厂合格率80\%.\\
        \(\frac{48}{x\%}=\frac{45}{(x-5)\%}\)
\end{solution}
\item 某市从今年\( 1 \)月\( 1 \)日起调整居民用水价格,每立方米水费上涨\( \frac{1}{3}\). 小丽家去年\( 12 \)月份的水费是\( 15 \)元,而今年\( 7 \)月份的水费则是\( 30 \)元.已知小丽家今年\( 7 \)月份的用水量比去年\( 12 \)月份的用水量多\( 5 m^3 \),求该市今年居民用水的价格.
\begin{solution}
        设该市去年居民用水的价格为\( x \)元\(/m^3\),则今年的水价为\((1 + \frac{1}{3}) x\)元\(/m3\),根据题意,得\\
        \(\frac{30}{(1+\frac{1}{3})x}-\frac{15}{x}=5\)\\
\(x = 3\)
经检验,\( x = \frac{3}{2}\)是所列方程的根.
\(\frac{3}{2} \times  (1 + \frac{1}{3}) \)\( = 2\)(元\(/m3\)).
所以,该市今年居民用水的价格为\( 2 \)元\(/m3.\)
\end{solution}
\item 小明和同学一起去书店买书,他们先用\( 15 \)元买了一种科普书,又用\( 15 \)元买了一种文学书.科普书的价格比文学书高出一半,他们所买的科普书比所买的文学书少\( 1 \)本.这种科普书和这种文学书的价格各是多少?
\begin{solution}
        文学书5元,科普书7.5元.\\
        \(\frac{15}{x}=\frac{15}{1.5x}+1\)
\end{solution}
\item 甲种原料与乙种原料的单价比为 2∶ 3, 将价值 2 000 元的甲种原料与价值 1 000 元的乙 种原料混合后, 单价为 9 元, 求甲种原料的单价.
\begin{solution}
        甲单价8元. \(\frac{2000}{2x}+\frac{1000}{3x}=\frac{2000+1000}{9}\)
\end{solution}
\item 某商店销售一批服装, 每件售价 150 元, 可获利 25\%. 求这种服装的成本价.
\begin{solution}
        成本价120元.\(\frac{150-x}{x}=25\%\)
\end{solution}
\item 某商店甲种糖果的单价为 20 元/kg, 乙种糖果的单价为 16 元/kg. 为了促销, 现 将 10 kg 乙种糖果和一包甲种糖果混合后( 搅匀) 销售, 如果将混合后的糖果单价定 为 17.5 元/kg, 那么混合销售与分开销售的销售额相同. 这包甲种糖果有多少千克?
\begin{solution}
        6kg. \(\frac{20x+16\times 10}{x+10}17.5\)
\end{solution}
\end{shiti}

\begin{verbatim}
\end{verbatim}

\subsubsection{备考练习}%
(时间:40分钟)
\begin{shiti}
\item 基础过关
    \begin{shiti}
    \item 解分式方程\(\frac{1-x}{x-2}=\frac{1}{2-x}-2\)时,去分母变形正确的是(\qquad)\\
        \begin{tasks}(1)
            \task \(-1+x= -1-2( x - 2)\)
            \task \(1-x = 1-2(x-2)\)
            \task \(-1+x = 1+2(2-x)\)
            \task \(1-x= -1-2(x-2)\)
        \end{tasks}
\begin{solution}
            D
\end{solution}
    \item 关于\(x\)的分式方程\(\frac{2}{x}+\frac{3}{x-a}= 0\)的解\(x=4\),则常数\(a\)的值为(\qquad)\\
        \begin{tasks}(4)
            \task \(  a = 1  \)
            \task \(  a = 2  \)
            \task \(  a = 4  \)
            \task \(  a = 10\)
        \end{tasks}
\begin{solution}
            D
\end{solution}
    \item 分式方程\(\frac{x-5}{x-1} + \frac{2}{x} = 1\)的解为(\qquad)\\
        \begin{tasks}(4)
            \task \(x=-1\)
            \task \(x=1\)
            \task \(x=2\)
            \task \(x=-2\)
        \end{tasks}
\begin{solution}
            A
\end{solution}
    \item 现代互联网技术的广泛应用,催生了快递行业的高速发展.据调查,湘潭某家小型快递公司的分拣工小李和小江,在分拣同一类物件时,小李分拣\(120\)个物件所用的时间与小江分拣\(90\)个物件所用的时间相同,已知小李每小时比小江多分拣\(20\)个物件,若设小江每小时分拣\(x\)个物件,则可列方程为(\qquad)\\
        \begin{tasks}(4)
            \task \(\frac{120}{x-20}=\frac{90}{x}\)
            \task \(\frac{120}{x+20}=\frac{90}{x}\)
            \task \(\frac{120}{x}=\frac{90}{x-20}\)
            \task \(\frac{120}{x}=\frac{90}{x+20}\)
        \end{tasks}
\begin{solution}
            B
\end{solution}
    \item 十堰即将跨入高铁时代.钢轨铺设任务也将完成.现还有\(6000\)米的钢轨需要铺设,为确保年底通车,如果实际施工时每天比原计划多铺设\(20\)米,就能提前\(15\)天完成任务,设原计划每天铺设钢轨\(*\)米,则根据题意所列的方程是(\qquad)\\
        \begin{tasks}(2)
            \task \(\frac{6000}{x}-\frac{6000}{x+20}=15\)
            \task \(\frac{6000}{x+20}-\frac{6000}{x}=15\)
            \task \(\frac{6000}{x}-\frac{6000}{x-15}=20\)
            \task \(\frac{6000}{x-15}-\frac{6000}{x}=20\)
        \end{tasks}
\begin{solution}
            A
\end{solution}
    \item 分式方程\(\frac{1}{x}=\frac{2}{x+1}\)的解为\(x =\)\tkt{\(1\)}
    \item 甲、乙两地相距\(1000 km\),如果乘高铁列车从甲地到乙地比乘特快列车少用\(3 h.\)已知高铁列车的平均速度是特快列车的\(1.6\)倍,设特快列车的平均速度为\(x \kilo\metre\per\hour \),根据题意可列方程为\tkt{\(\frac{1000}{x}-\frac{1000}{1.6x}=3\)}  .
    \item 若关于\(x\)的分式方程\(\frac{3x}{x-2}-1=\frac{m+3}{x-2}\)有增根,则\(m\)的值为\tkt{\(3\)} .
    \item 解方程:\(\frac{x-2}{x-3}+1=\frac{2}{3-x}\).
\begin{solution}
            \(x=\frac{3}{2}\)
\end{solution}
    \item 解分式方程:\(\frac{x}{x-2}-1=\frac{4}{x^2-4x+4}\)
\begin{solution}
            \(x=4\)
\end{solution}
    \item 某公司购买了一批A、B型芯片,其中A型芯片的单价比B型芯片的单价少\(9\)元,已知该公司用\(3120\)元购买\(A\)型芯片的条数与用\(4200\)元购买\(B\)型芯片的条数相等.\\
        (1)求该公司购买的\(A、B\)型芯片的单价各是多少元?\\
        (2)若两种芯片共购买了\( 200\)条,且购买的总费用为\(6280\)元,求购买了多少条\(A\)型芯片?
\begin{solution}
            (1) A:26元; B:35元.\\
            (2) 80 条.
\end{solution}
    \end{shiti}
\item 提高
    \begin{shiti}[resume]
    \item 若关于\(x\)的分式方程\(\frac{x}{x-3}+\frac{3a}{3-x}=2a\)无解,则\(a\)的值为\tkt{\(1\text{或}\frac{1}{2}\)} .
    \item 在我市``青山绿水''行动中,某社区计划对面积有\(3600 \squaremetre \)的区域进行绿化,经投标由甲、乙两个工程队来完成.已知甲队每天能完成绿化的面积是乙队每天能完成绿化面积的\(2\)倍,如果两队各自独立完成面积为\(600 m2\)区域的绿化时,甲队比乙队少用\(6\)天.\\
        (1)求甲、乙两工程队每天各能完成多少面积的绿化;\\
        (2)若甲队每天绿化费用是1.2万元.乙队每天绿化费用为0.5万元,社区要使这次绿化的总费用不超过40元,则至少应安排乙工程队绿化多少天?
\begin{solution}
            (1) 甲100,乙50 ;\\
            (2) 32天
\end{solution}
    \end{shiti}
\item 核心素养
    \begin{shiti}
    \item 在求\(3x\)的倒数的值时,嘉淇同学误将\(3x\)看成了\(8x\),她求得的值比正确答案小5.依上述情形,所列关系成立的是(\qquad)\\
    \begin{tasks}(4)
    \task \(\frac{1}{3x}=\frac{1}{8x}-5\)
    \task \(\frac{1}{3x}=\frac{1}{8x}+5\)
    \task \(\frac{1}{3x}=8x-5\)
    \task \(\frac{1}{3x}=8x+5\)
    \end{tasks}
\begin{solution}
        B
\end{solution}
    \item 某工厂计划生产\(1500\)个零件,但是在实际生产时,\(  \dots \),求实际每天生产零件的个数在这个题目中,若设实际每天生产零件\(x\)个,可得方程\(\frac{1500}{x-5}-\frac{1500}{x}=10\).则题目中用\(`` \dots '' \)表示的条件应是(\qquad)\\
    \begin{tasks}(1)
        \task 每天比原计划多生产\(5\)个,结果延期\(10\)天完成
        \task 每天比原计划多生产\(5\)个,结果提前\(10\)天完成
        \task 每天比原计划少生产\(5\)个,结果延期\(10\)天完成
        \task 每天比原计划少生产\(5\)个,结果提前\(10\)天完成
    \end{tasks}
\begin{solution}
        B
\end{solution}
    \end{shiti}
\end{shiti}

\end{document}
