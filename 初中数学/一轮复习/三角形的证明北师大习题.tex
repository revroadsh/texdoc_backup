\documentclass[cn,blue,12pt]{elegantbook}
\input {d:/tex/preamble}

%\excludecomment{note}
%\excludecomment{solution}
%\renewcommand \tkt[1]{{\CJKunderline[hidden=true, skip=true, thickness=1pt]{#1}}}

\begin{document}
\begin{liti}[resume]
\item \textbf{定理:等边对等角的证明}.在\( \triangle ABC \)中,\( AB = AC \). 求证:\( \angle B = \angle C\). 
\textbf{推论等腰三角形顶角的平分线、底边上的中线及底边上的高线互相重合.} 
\begin{solution}
证明:取\( BC \)的中点\( D\),连接\( AD\). 
\( \because AB = AC\),\( BD = CD\),\( AD = AD\),
\( \therefore \triangle ABD \cong  \triangle ACD( SSS) \). 
\( \therefore \angle B = \angle C \)(全等三角形的对应角相等). 
\end{solution}
\item 如图,在\( \triangle ABD \)中,\( AC \perp BD\),垂足为\( C\),\( AC = BC = CD\). \\
    \begin{liti}[resume]
    \item 求证:\( \triangle ABD \)是等腰三角形;\\
    \item \(( 2) \)求\( \angle BAD \)的度数. \\
    \end{liti}
\item 将下面证明中每一步的理由写在括号内. 
已知:如图,\( AB = CD\),\( AD = CB\). 
求证:\( \angle  A =  \angle  C\). 
证明:如图,连接\( BD\). 
在\( \triangle BAD \)和\( \triangle DCB \)中,
\( \because AB = CD( )\),
\(AD = CB( )\),
\(BD = DB( )\),
\( \therefore \triangle BAD \cong  \triangle DCB( ) \). 
\( \therefore \angle A = \angle C( ) \). 
\item 已知:如图,点\(B\),\( E\),\( C\),\( F\)在同一条直线上,\( AB = DE\),
\(AC = DF\),\( BE = CF\). 
求证:\( \angle A = \angle D\). 
\item 如图,在\( \triangle ABC \)中,\( \angle BAC = 108 ^\circ \),\( AB = AC\),\( AD \perp \)
\(BC\)
,垂足为\( D\),求\( \angle  BAD \)的度数. 
数学理解
\item 如图,在\( \triangle ABC \)中,\( AB = AC\),\( AD \perp BC\),垂足为\( D\),
点\( E \)是\( AD \)上一点,连接\( BE\),\( CE\). 请找出图中所有相等的
角,并说明理由. 
\item 两个等腰三角形的顶角和底边分别相等,那么这两个三角形全等吗?
. 请证明你的结论. 
问题解决
\item 如图,在\( \triangle ABC \)中,\( AB = AC\),点\( D\),\( E \)都在边\( BC \)上,
且\( AD = AE\),那么\( BD \)与\( CE \)相等吗?请证明你的结论. 
证明:等腰三角形两底角的平分线相等. 
已知:如图\( 1-4\),在\( \triangle ABC \)中,\( AB = AC\),\( BD \)和\( CE \)是\( \triangle ABC \)的角平分线. 
求证:\( BD = CE\). 
证明:\( \because AB = AC\),
\( \therefore \angle ABC = \angle ACB( \)等边对等角.. 
\( \because BD\),\( CE \)分别平分\( \angle ABC \)和\( \angle ACB\),
\( \therefore \angle  1 = 1\)
\item \( \angle  ABC\),\( \angle  2 = 21 \angle  ACB\). 
\( \therefore \angle  1 =  \angle  2\). 
在\( \triangle BDC \)和\( \triangle CEB \)中,
\( \because \angle  ACB =  \angle  ABC\),\( BC = CB\),\( \angle  1 =  \angle  2\),
\( \therefore \triangle BDC \cong  \triangle CEB( ASA) \). 
\( \therefore BD = CE( \)全等三角形的对应边相等.. 
求等边三角形两条中线相交所成锐角的度数. 
\item 如图,在\( \triangle ABC \)中,\( D\),\( E \)是\( BC \)的三等分
点,且\( \triangle ADE \)是等边三角形,求\( \angle BAC \)的
度数. 
\(A\)
\(B C\)
\(D \)
\item 如图,在\( \triangle ABC \)中,\( AB = AC\),\( BD \)平分\( \angle ABC\),交\( AC \)于点\task 若\( BD = BC\),
则\( \angle A \)等于多少度?
\item 已知:如图,在\( \triangle ABC \)中,\( AB = AC\),\( D \)为\( BC \)的中点,点\( E\),\( F \)分别在\( AB \)和\( AC \)上,
并且\( AE = AF\). 求证:\( DE = DF\). 
.第\( 1 \)题\() ( \)第\( 2 \)题\() ( \)第\( 3 \)题.
\(B C\)
\(D\)
\(E F\)
\(A\)
\(B C\)
\(D\)
\(E\)
\(A\)
\(A\)
\(B C\)
\(D\)
\item 已知:如图,点\( D\),\( E \)分别是等边三角形\( ABC \)的两边\( AB\),\( AC \)上的点,且\( AD = CE\). 
求证:\( CD = BE\). 
数学理解
\item 如图,在一个风筝\( ABCD \)中,\( AB = AD\),\( BC = DC\). 
\(( 1) \)分别在\( AB\),\( AD \)的中点\( E\),\( F \)处拉两根彩线\( EC\),\( FC\),证明:
这两根彩线的长度相等;
\(( 2) \)如果\( AE = 1\)
\item \( AB\),\( AF = 1 3 AD\),那么彩线的长度相等吗?
如果\( AE = 1\)
\item \( AB\),\( AF = 1 4 AD \)呢?由此你能得到什么结论?
\(( 3) \)除了\(( 1)( 2) \)的条件外,你还能在哪些已知条件下得到两
根彩线的长度相等的结论
已知:如图\( 1-8\),\( AB = DC\),\( BD = CA\). 
求证:\( \triangle AED \)是等腰三角形. 
证明:\( \because AB = DC\),\( BD = CA\),\( AD = DA\),
\( \therefore \triangle ABD \cong  \triangle DCA( SSS) \). 
\( \therefore \angle ADB = \angle DAC( \)全等三角形的对应角
相等.. 
\( \therefore AE = DE( \)等角对等边.. 
\( \therefore \triangle AED \)是等腰三角形
用反证法证明:一个三角形中不能有两个角是直角. 
已知:\( \triangle ABC\). 
求证:\( \angle A\),\( \angle B\),\( \angle C \)中不能有两个角是直角. 
证明:假设\( \angle A\),\( \angle B\),\( \angle C \)中有两个角是直角,不妨设\( \angle A \)和\( \angle B \)是
直角,即\( \angle A = 90 ^\circ \),\( \angle B = 90 ^\circ \). 
于是\( \angle A + \angle B + \angle C = 90 ^\circ + 90 ^\circ + \angle C > 180 ^\circ \). 
这与三角形内角和定理相矛盾,因此``\( \angle A \)和\( \angle B \)是直角'' 的假设不成立. 
所以,一个三角形中不能有两个角是直角
在一个三角形中,如果两个角不相等,那么这两个角所对的边
也不相等. 你认为小明这个结论成立吗?如果成立,你能证明它吗?
小明是这样想的:
\(A\)
\(B C\)
图\( 1-9\)
如图\( 1-9\),在\( \triangle ABC \)中,已
知\( \angle B \ne \angle  C\),此时\( AB \)与\( AC \)要
么相等,要么不相等. 
假设\( AB = AC\),那么根据``等
边对等角'' 定理可得\( \angle C = \angle B\),
这与已知条件\( \angle B \ne \angle C \)相矛盾,因此\( AB \ne AC\). 
\item 如图,在\( \triangle ABC \)中,\( BD \)平分\( \angle ABC\),交\( AC \)于点\( D\),
过点\( D \)作\( DE \parallel BC\),交\( AB \)于点\( E\),请判断\( \triangle BDE \)的形
状,并说明理由. 
\item 已知五个正数的和等于\( 1\),用反证法证明:这五个数中
至少有一个大于或等于\( 1 5 \). 
\item 已知:如图,\( \angle  CAE \)是\( \triangle ABC \)的外角,\( AD \parallel BC\),
且\(  \angle  1 =  \angle  2\). 
求证:\( AB = AC\). 
例\(3\)
.第\( 1 \)题.
\(A\)
\(B C\)
\(D\)
\(E 1210\)
数学\(  \)八年级\(  \)下册
\item 已知:如图,在\( \triangle ABC \)中,\( AB = AC\),点\( E \)在\( CA \)的延长线上,
\(EP \perp BC\),垂足为\( P\),\( EP \)交\( AB \)于点\( F\). 求证:\( \triangle AEF \)是等腰三
角形. 
数学理解
\item \( ( 1) \)已知:如图.甲.,等腰三角形的一个内角为锐角\( \alpha \),腰
为\( a\),求作这个等腰三角形;
\( \alpha a\)
.甲.
\( \alpha (\)
乙.
.第\( 3 \)题.
\(( 2) \)在\(( 1) \)中,把锐角\( \alpha  \)变成钝角\( \alpha \),其他条件不变,求作这个等腰三角形. 
问题解决
\item 如图,一艘船从\( A \)处出发,以\( 18 kn1 \)的速度向正北航行,经
过\( 10 h \)到达\( B \)处. 分别从\( A\),\( B \)望灯塔\( C\),测得\( \angle NAC = 42 ^\circ \),
\( \angle NBC = 84 ^\circ \). 求从\( B \)处到灯塔\( C \)的距离. 
在直角三角形中,如果一个锐角等于\(30 ^\circ \),那么它所对的直角
边等于斜边的一半. 
已知:如图\( 1-10( 1)\),\( \triangle ABC \)是直角三角形,\( \angle  C = 90 ^\circ \),\( \angle  A = 30 ^\circ \). 
求证:\( BC = 1\)
\item \( AB\). 
图\( 1-10\)
\(( 1)\)
\(A\)
\(B C\)
\(( 2)\)
\(A\)
\(B D\)
\(C\)
证明:如图\( 1-10( 2)\),延长\( BC \)至点\( D\),使\( CD = BC\),连接\( AD\). 
\( \because \angle ACB = 90 ^\circ \),\( \angle BAC = 30 ^\circ \),
\( \therefore \angle ACD = 90 ^\circ \),\( \angle B = 60 ^\circ \). 
\( \because AC = AC\),
\( \therefore \triangle ABC \cong  \triangle ADC( SAS) \). 
\( \therefore AB = AD( \)全等三角形的对应边相等.. 
\( \therefore \triangle ABD \)是等边三角形.有一个角等于\(60 ^\circ \)的等腰三角形是等边三
角形.. 
\( \therefore BC = 1\)
\item \( BD = 21 AB\). 
求证:如果等腰三角形的底角为\( 15 ^\circ \),那么腰上的高是腰长的一半. 
已知:如图\( 1-11\),在\( \triangle ABC \)中,\( AB = AC\),\( \angle  B =15 ^\circ \),\( CD \)是腰\( AB \)上
例\(312\)
数学\(  \)八年级\(  \)下册
的高. 
求证:\( CD = 1\)
\item \( AB\). 
证明:在\( \triangle ABC \)中,
\( \because AB = AC\),\( \angle  B = 15 ^\circ \),
\( \therefore \angle ACB = \angle B = 15 ^\circ ( \)等边对等角.. 
\( \therefore \angle DAC = \angle B + \angle ACB = 15 ^\circ + 15 ^\circ =30 ^\circ \). 
\( \because CD \)是腰\( AB \)上的高,
\( \therefore \angle ADC = 90 ^\circ \). 
\( \therefore CD = 1\)
\item \( AC( \)在直角三角形中,如果一个锐角等于\(30 ^\circ \),那么它所对
的直角边等于斜边的一半.. 
\( \therefore CD = 1\)
\item \( AB\). 
随堂练习
直角三角形的一个角等于\(30 ^\circ \),斜边长为\( 4\),用四个这样的直角三角形拼成如图所示
的正方形,求正方形\( EFGH \)的边长. 
\item 已知:如图,\( \triangle ABC \)是等边三角形,\( DE \parallel BC\),分别交\( AB \)和\( AC \)于点\( D\),\( E\). 
图\( 1-11\)
\(A\)
\(B C\)
\(D13\)
第一章\(  \)三角形的证明
求证:\( \triangle ADE \)是等边三角形. 
.第\( 1 \)题.
\(A\)
\(B C\)
\(D E\)
.第\( 2 \)题.
\(A C\)
\(B\)
\(D E\)
\item 房梁的一部分如图所示,其中\( BC \perp AC\),\( \angle A = 30 ^\circ \),\( AB = 7.4 m\),点\( D \)是\( AB \)的中
点,且\( DE \perp AC\),垂足为\( E\),求\( BC\),\( DE \)的长. 
联系拓广
\item \( ( 1) \)如图,\( \triangle ABC \)是等边三角形,过它的三个顶点分别作对
边的平行线,得到一个新的\( \triangle DEF\),\( \triangle DEF \)是等边三角
形吗?你还能找到其他的等边三角形吗?点\( A\),\( B\),\( C \)分
别是\( EF\),\( ED\),\( FD \)的中点吗?请证明你的结论. 
\(( 2) \)如果\( \triangle DEF \)是等边三角形,点\( A\),\( B\),\( C \)分别是\( EF\),
\(ED\)
,\( FD \)的中点,那么\( \triangle ABC \)是等边三角形吗?请证明
你的结论. 
\item 证明:在直角三角形中,如果一条直角边等于斜边的一半,那么这条直角边所对的锐
角等于\( 30 ^\circ \). 
\item 如图\(( 1)\),\( ABCD \)是一张正方形纸片,\( E\),\( F \)分别为\( AB\),\( CD \)的中点,沿过点\( D \)的
折痕将\( A \)角翻折,使得点\( A \)落在\( EF \)上.如图\(( 2) \)的点\( A′)\),折痕交\( AE \)于点\( G\),那
么\( \angle  ADG \)等于多少度?你能证明你的结论吗?.提示:利用第\( 4 \)题的结论
已知:如图\( 1-12( 1)\),在\( \triangle ABC \)中,\( AB2 + AC2 = BC2\). 
\(A\)
\(B C\)
\(A′\)
\(B′ C′\)
\(( 1) ( 2)\)
图\( 1-1215\)
第一章\(  \)三角形的证明
求证:\( \triangle ABC \)是直角三角形. 
证明:如图\( 1-12( 2)\),作\( Rt \triangle A′B′C′\),使
\(    \angle  A′ = 90 ^\circ \),\( A′B′ = AB\),\( A′C′ = AC\),
则\( A′B′2 + A′C′2 = B′C′2( \)勾股定理.. 
\( \because AB2 + AC2 = BC2\),
\( \therefore BC2 = B′C′2\). 
\( \therefore BC = B′C′\). 
\( \therefore \triangle ABC \cong  \triangle A′B′C(′ SSS) \). 
\( \therefore \angle  A = \angle A′ = 90 ^\circ ( \)全等三角形的对应角相等.. 
因此,\( \triangle ABC \)是直角三角形
\item 在\( \triangle ABC \)中,已知\( \angle A = \angle B = 45 ^\circ \),\( BC = 3\),求\( AB \)的长. 
\item 已知:在\( \triangle ABC \)中,\( AB = 13 cm\),\( BC = 10 cm\),\( BC \)边上的中线\( AD = 12 cm\). 
求证:\( AB = AC\). 
\item 说出下列命题的逆命题,并判断每对命题的真假:
\(( 1) \)四边形是多边形;
\(( 2) \)两直线平行,同旁内角互补;
\(( 3) \)如果\( ab = 0\),那么\( a = 0\),\( b = 0\). 
利用教科书给出的基本事实和已有定理,我们可以证明勾股定理. 
如图\( 1-13( 1)\),在\( \triangle ABC \)中,\( \angle  C = 90 ^\circ \),\( BC = a\),\( AC = b\),
\(AB = c\). 
分别以\( Rt \triangle ABC \)的三边为边长作正方形\( AHIB\),\( ACDE\),\( CBFG( \)如
图\( 1-13( 2) ) \). 连接\( EB\),\( CH\). 过点\( C \)作\( AB \)的垂线,分别交\( AB \)和\( HI \)于\(17\)
第一章\(  \)三角形的证明
图\( 1-13\)
\(A B\)
\(C\)
\(b a\)
\(c\)
\(( 1) ( 2)\)
\(A M B\)
\(D\)
\(E\)
\(F\)
\(G\)
\(H N I\)
\(C\)
\(b a\)
\(c\)
点\( M\),\( N\). 
\( \because EA = CA\),\( \angle EAB = \angle CAH = 90 ^\circ + \angle CAB\),\( AB = AH\),
\( \therefore \triangle EAB \cong   \triangle CAH (SAS )\). 
又\( \because S\)正方形\(ACDE = 2S \triangle EAB\),\( S\)长方形\(AHNM = 2S \triangle CAH\),
\( \therefore b2 = S\)长方形\(AHNM \). 
同理\( a2 = S\)长方形\(MNIB \). 
\( \therefore c2 = a2 + b2\). 
以上是欧几里得在\(《 \)原本\(》 \)中证明勾股定理的大致过程. 
勾股定理是数学史上非常重要的定理之一. 两千多年来,人们对它进
行了大量的研究,给出了多达数百种的证明方法. 如果你有兴趣,可查阅
有关资料,了解勾股定理的其他证明方法
\item 如图,在四边形\( ABCD \)中,\( AB \parallel CD\),\( E \)为\( BC \)上的一点,
且\( \angle BAE = 25 ^\circ \),\( \angle CDE = 65 ^\circ \),\( AE = 2\),\( DE = 3\),求\( AD \)的长. 
.第\( 1 \)题.
\(A\)
\(D\)
\(B\)
\(C\)
\(E18\)
数学\(  \)八年级\(  \)下册
\item 一个直角三角形房梁如图所示,其中\( BC \perp AC\),\( \angle A = 30 ^\circ \),\( AB = 10 m\),\( CB1 \perp AB\),
\(B\)
\item \( 1 \perp AC\),垂足分别为\( B1\),\( C1\),那么\( BC \)的长是多少?\( B1C1 \)呢?
\(A\)
\(B\)
\(C\)
\(C1\)
\(B\)
\(1\)
.第\( 2 \)题.
问题解决
\item 如图,小红想测量离\( A \)处\( 30 m \)的大树的高度,她站在\( A \)处仰望树顶\( B\),仰角为\( 30 ^\circ \)
.即\( \angle BDE = 30 ^\circ ) \). 已知小红身高\( 1.52 m\),求大树的高度.结果精确到\( 0.1 m) \). 
\(B\)
\(E 30 ^\circ \)
\(D A\)
.第\( 3 \)题.
\(A\)
\(B\)
\(D C\)
\(A′ B′\)
\(C′\)
\(D′\)
.第\( 5 \)题.
\item 有一块三角形空地,它的三条边线分别长\( 45 m\),\( 60 m \)和\( 70 m\). 已知\( 60 m \)长的边线为
南北向,是否有一条边线为东西向?
\item 如图,正四棱柱的底面边长为\( 5 cm\),侧棱长为\( 8 cm\),一只蚂蚁从点\( A \)出发,沿棱柱
侧面到点\( C′ \)处吃食物,那么它需要爬行的最短路径的长是多少?
斜边和一条直角边分别相等的两个直角三角形全等. 
已知:如图\( 1-15\),在\( \triangle ABC \)与\( \triangle A′B′C′ \)中,\( \angle  C = \angle C′ = 90 ^\circ \),\( AB = A′B′\),
\(AC =A′C′\). 
求证:\( \triangle ABC \cong   \triangle A′B′C′\). 
证明:在\( \triangle ABC \)中,
\( \because \angle  C = 90 ^\circ \),
\( \therefore BC2 = AB2 - AC2( \)勾股定理.. 
同理,\( B′C′2 = A′B′2 - A′C′2\). 
\( \because AB = A′B′\),\( AC = A′C′\),
\( \therefore BC = B′C′\). 
\( \therefore \triangle ABC \cong   \triangle A′B′C(′ SSS) \). 
\(A\)
\(C B\)
\(A′\)
\(C′ B′\)
图\( 1-15\)
\(a\)
\(c \alpha \)
图\( 1-1420\)
数学\(  \)八年级\(  \)下册
这一定理可以简述为``斜边、直角边'' 或``\( HL\)'' . 
如图\( 1-16\),有两个长度相等的滑梯,左边滑梯的高度\( AC \)与右边滑
梯水平方向的长度\( DF \)相等,两个滑梯的倾斜角\( \angle  B \)和\( \angle  F \)的大小有什么
关系?
解:根据题意,可知
\(    \angle  BAC = \angle EDF = 90 ^\circ \),
\(BC = EF\),\( AC = DF\),
\( \therefore Rt \triangle BAC \cong  Rt \triangle EDF( HL) \). 
\( \therefore \angle  B = \angle  DEF( \)全等三角形的对应角
相等.. 
\( \because \angle  DEF + \angle  F = 90 ^\circ ( \)直角三角形的
两锐角互余.,
\( \therefore \angle  B + \angle  F = 90 ^\circ \). 
随堂练习
\item 判断下列命题的真假,并说明理由:
\(( 1) \)两个锐角分别相等的两个直角三角形全等;
\(( 2) \)斜边及一锐角分别相等的两个直角三角形全等;
\(( 3) \)两条直角边分别相等的两个直角三角形全等;
\(( 4) \)一条直角边相等且另一条直角边上的中线相等的
两个直角三角形全等. 
\item 如图,两根长度为\( 12 m \)的绳子,一端系在旗杆上,
另一端分别固定在地面的两个木桩上,两个木桩离旗
杆底部的距离相等吗?请说明你的理由. 
\(A\)
\(B O C\)
.第\( 2 \)题.
例
\(A D\)
\(E\)
\(F\)
\(C\)
\(B\)
图\( 1-1621\)
第一章\(  \)三角形的证明
习题\( 1.6\)
知识技能
\item 判断下列命题的真假,并说明理由:
\(( 1) \)两边分别相等的两个直角三角形全等;
\(( 2) \)一个锐角和一条边分别相等的两个直角三角形全等. 
\item 已知:如图,\( D \)是\( \triangle ABC \)的\( BC \)边的中点,\( DE \perp AC\),\( DF \perp AB\),垂足分别为\( E\),\( F\),
且\( DE = DF\). 求证:\( \triangle ABC \)是等腰三角形. 
\(A B\)
\(D\)
\(E F\)
\(C\)
.第\( 3 \)题.
\(A\)
\(C\)
\(D\)
\(F E\)
\(B\)
.第\( 2 \)题.
\item 已知:如图,\( AB = CD\),\( DE \perp AC\),\( BF \perp AC\),垂足分别为\( E\),\( F\),且\( DE = BF\). 求证:
\(( 1) AE = CF\);\( ( 2) AB \parallel CD\). 
问题解决
\item 用三角尺可以画角平分线:如图所示,在已知\( \angle  AOB \)的
两边上分别取点\( M\),\( N\),使\( OM = ON\),再过点\( M \)画\( OA \)的
垂线,过点\( N \)画\( OB \)的垂线,两垂线交于点\( P\),那么射
线\( OP \)就是\( \angle  AOB \)的平分线. 请你证明这一结论. 
联系拓广
\item 在如图所示的三角形纸片\( ABC \)中,\( \angle  C = 90 ^\circ \),\( \angle  B = 30 ^\circ \). 按如下步骤可以把这个直角三
角形纸片分成三个全等的小直角三角形.图中虚线表示折痕.:\ding{172}折叠三角形纸片\( ABC\),
使点\( B \)与点\( A \)重合;\ding{173}将折叠后的纸片再沿\( AD \)对折. 
\(( 1) \)由步骤\ding{172}可以得到哪些等量关系?
\(( 2) \)请证明\( \triangle ACD \cong   \triangle AED\);
\(( 3) \)按照这种方法能否将任意一个直角三角形分成三个全
等的小三角形
线段垂直平分线上的点到这条线段两个端点的距离相等. 
已知:如图\( 1-17\),直线\( MN \perp AB\),垂足为\( C\),且\( AC = BC\),\( P \)是\( MN \)上
的任意一点. 
求证:\( PA = PB\). 
证明:\( \because MN \perp AB\),
\( \therefore \angle  PCA =  \angle  PCB = 90 ^\circ \). 
\( \because AC = BC\),\( PC = PC\),
\( \therefore \triangle PCA \cong  \triangle PCB( SAS) \). 
\( \therefore PA = PB( \)全等三角形的对
应边相等.. 
已知:如图\( 1-18\),在\( \triangle ABC \)中,\( AB = AC\),\( O \)是\( \triangle ABC \)内一点,
且\( OB = OC\). 
求证:直线\( AO \)垂直平分线段\( BC\). 
证明:\( \because AB = AC\),
\( \therefore \)点\( A \)在线段\( BC \)的垂直
平分线上.到一条线段两个端点
距离相等的点,在这条线段的垂
直平分线上.. 
同理,点\( O \)在线段\( BC \)的垂
直平分线上. 
\( \therefore \)直线\( AO \)是线段\( BC \)的垂直平分线.两点确定一条直线.. 
已知:如图,\( AB \)是线段\( CD \)的垂直平分线,\( E\),\( F \)是\( AB \)上的两点. 
求证:\( \angle  ECF = \angle  EDF\). 
如图,在\( \triangle ABC \)中,\( AB = AC\),\( \angle  BAC = 120 ^\circ \),\( AB \)的
垂直平分线交\( AB \)于点\( E\),交\( BC \)于点\( F\),连接\( AF\),
求\( \angle  AFC \)的度数. 
数学理解
\item 在以线段\( AB \)为底边的所有等腰三角形中,它们另一个顶点的位置有什么共同特征?
你还有其他
证明方法吗?
\(A\)
\(B C\)
\(O\)
图\( 1-18\)
.第\( 1 \)题.
\(A\)
\(E F\)
\(B C24\)
数学\(  \)八年级\(  \)下册
问题解决
\item 如图,在\( \triangle ABC \)中,已知\( AC = 27\),\( AB \)的垂直平分线交\( AB \)于点\( D\),交\( AC \)于点\( E\),
\( \triangle BCE \)的周长等于\( 50\),求\( BC \)的长. 
\(A\)
\(B C\)
\(D\)
\(E\)
.第\( 3 \)题\() ( \)第\( 4 \)题.
\(A\)
\(B\)
\item 如图,\( A\),\( B \)表示两个仓库,要在\( A\),\( B \)一侧的河岸边建造一个码头,使它到两个仓库
的距离相等,码头应建造在什么位置?
求证:三角形三条边的垂直平分线相交于一
点,并且这一点到三个顶点的距离相等. 
已知:如图\( 1-19\),在\( \triangle ABC \)中,边\( AB \)的垂直平
分线与边\( BC \)的垂直平分线相交于点\( P\). 
求证:边\( AC \)的垂直平分线经过点\( P\),且\( PA =\)
\(PB = PC\). 
证明:\( \because \)点\( P \)在线段\( AB \)的垂直平分线上,
\( \therefore PA = PB( \)线段垂直平分线上的点到这条线段两个端点的距离相等.. 
同理,\( PB = PC\). 
\( \therefore PA = PB = PC\). 
\( \therefore \)点\( P \)在线段\( AC \)的垂直平分线上.到一条线段两个端点距离相等的
点,在这条线段的垂直平分线上.,
即边\( AC \)的垂直平分线经过点\( P\)
在\( \triangle ABC \)中,\( BC = 2\),\( \angle  BAC > 90 ^\circ \),\( AB \)的垂直平分线交\( BC \)于点\( E\),
\(AC \)的垂直平分线交\( BC \)于点\( F\),请找出图中相等的线段,并求\( \triangle AEF \)的周长. 
\item 如图,已知线段\( a\),求作以\( a \)为底、以\( 1\)
\item \( a \)为高的等腰三角形,这个等腰三角形
有什么特征?
.第\( 2 \)题.
\(A\)
\(C B\)
\(a\)
.第\( 1 \)题.
\item 如图,已知\( \triangle ABC\),求作:
\(( 1) AC \)边上的高;\( ( 2) BC \)边上的高. 
问题解决
\item 为筹办一个大型运动会,某市政府打算修建一个大型体育中心. 在选址过程中,有人建议
该体育中心所在位置应与该市的三个城镇中心.图中以\( P\),\( Q\),\( R \)表示.的距离相等\(.27\)
第一章\(  \)三角形的证明
\(( 1) \)根据上述建议,试在图\(( 1) \)中画出体育中心\( G \)的位置;
\(P\)
\(P\)
\(Q\)
\(R Q\)
\(R\)
\(( 1) ( 2)\)
.第\( 3 \)题.
\(( 2) \)如果这三个城镇的位置如图\(( 2) \)所示,\( \angle  RPQ \)是一个钝角,那么根据上述建议,
体育中心\( G \)应在什么位置?
\(( 3) \)你对上述建议有何评论?你对选址有什么建议?
\item 如图,某市三个城镇中心\( A\),\( B\),\( C \)恰好分别位于一个等边三角形的三个顶点处,在三
个城镇中心之间铺设通信光缆,以城镇\( A \)为出发点设计了三种连接方案:
\(( 1) AB + BC\);
\(( 2) AD + BC( D \)为\( BC \)的中点.;
\(( 3) OA + OB + OC( O\)为\( \triangle ABC \)三边的垂直平分线的交点.. 
要使铺设的光缆长度最短,应选哪种方案
定理角平分线上的点到这个角的两边的距离相等. 
已知:如图\( 1-22\),\( OC \)是\( \angle  AOB \)的平分线,点\( P \)在\( OC \)上,\( PD \perp OA\),
\(PE \perp OB\),垂足分别为\( D\),\( E\). 
求证:\( PD = PE\). 
证明:\( \because PD \perp OA\),\( PE \perp OB\),垂足分别为\( D\),\( E\),
\( \therefore \angle  PDO = \angle  PEO = 90 ^\circ \). 
\( \because \angle  1 = \angle  2\),\( OP = OP\),
\( \therefore \triangle PDO \cong   \triangle PEO( AAS) \). 
\( \therefore PD = PE( \)全等三角形的对应边相等.. 
定理在一个角的内部,到角的两边距离相等的点在这个角的平分
线上. 
已知:如图\( 1-23\),点\( P \)为\( \angle  AOB \)内一点,\( PD \perp OA\),
\(PE \perp OB\),垂足分别为\( D\),\( E\),且\( PD = PE\). 
求证:\( OP \)平分\( \angle  AOB\). 
证明:\( \because PD \perp OA\),\( PE \perp OB\),垂足分别为\( D\),\( E\),
\( \therefore \angle  ODP = \angle  OEP = 90 ^\circ \). 
\( \because PD = PE\),\( OP = OP\),
\( \therefore Rt \triangle DOP \cong  Rt \triangle EOP( HL) \). 
\( \therefore \angle  1 = \angle 2( \)全等三角形的对应角相等.. 
\( \therefore OP \)平分\( \angle AOB\). 
如图\( 1-24\),在\( \triangle ABC \)中,\( \angle  BAC = 60 ^\circ \),点\( D \)在\( BC \)上,\( AD = 10\),
\(DE \perp AB\),\( DF \perp AC\),垂足分别为\( E\),\( F\),且\( DE = DF\),求\( DE \)的长. 
解:\( \because DE \perp AB\),\( DF \perp AC\),垂足分别为\( E\),\( F\),
且\( DE = DF\),
\( \therefore AD \)平分\( \angle  BAC( \)在一个角的内部,到角的两边
距离相等的点在这个角的平分线上.. 
又\( \because \angle  BAC = 60 ^\circ \),
\( \therefore \angle  BAD = 30 ^\circ \). 
在\( Rt \triangle ADE \)中,\( \angle  AED = 90 ^\circ \),\( AD = 10\),
\( \therefore DE= 1\)
\item \( AD = 21 \times 10 = 5( \)在直角三角形中,如果一个锐角等于\(30 ^\circ \),
那么它所对的直角边等于斜边的一半.. 
随堂练习
\item 如图,\( AD\),\( AE \)分别是\( \triangle ABC \)中\( \angle  A \)的内角平分线和外角平分线,它们有什么
关系?
.第\( 1 \)题.
\(B A\)
\(C\)
\(D\)
\(E\)
\(F\)
.第\( 2 \)题.
\(A \)区
\item 如图,一目标在\( A \)区,到公路、铁路距离相等,离公路与铁路交叉处\( 500 m\),在
图上标出它的位置.比例尺\( 1∶ 20 000)\)
\item 已知:如图,在\( \triangle ABC \)中,\( AD \)是它的角平分线,且\( BD = CD\),
\(DE \perp AB\),\( DF \perp AC\),垂足分别为\( E\),\( F\). 
求证:\( EB = FC\). 
联系拓广
\item 如图,在\( \triangle ABC \)中,\( \angle  C = 90 ^\circ \),\( \angle  A = 30 ^\circ \),作\( AB \)的垂直平分线,交\( AB \)于点\( D\),
交\( AC \)于点\( E\),连接\( BE\),则\( BE \)平分\( \angle  AB \)\task 请证明这一结论. 你有几种证明方法?
\(A\)
\(B C\)
\(E\)
\(D\)
.第\( 3 \)题.
\(A\)
\(B\)
\(C\)
\(O\)
\(D\)
.第\( 4 \)题.
\item 如图,求作一点\( P\),使\( PC = PD\),并且点\( P \)到\( \angle  AOB \)两边的距离相等. 
求证:三角形的三条角平分线相交于一点,并且这一点到三条边的距
离相等. 
已知:如图\( 1-25\),在\( \triangle ABC \)中,角平分线\( BM \)与角平分线\( CN \)相交于
点\( P\),过点\( P \)分别作\( AB\),\( BC\),\( AC \)的垂线,垂足分别是\( D\),\( E\),\( F\). 
求证:\( \angle  A \)的平分线经过点\( P\),且\( PD = PE = PF\). 
证明:\( \because BM \)是\( \triangle ABC \)的角平分线,点\( P \)在\( BM \)上,
\( \therefore PD = PE( \)角平分线上的点到这个角的两边的距
离相等.. 
同理,\( PE = PF\). 
\( \therefore PD = PE = PF\). 
\( \therefore \)点\( P \)在\( \angle  A \)的平分线上.在一个角的内部,到
例\(2\)
\(A\)
\(B C\)
\(D\)
\(E F\)
.第\( 2 \)题.
\(A\)
\(B C\)
\(E\)
\(D\)
\(P M\)
\(F\)
\(N\)
图\( 1-2531\)
第一章\(  \)三角形的证明
角的两边距离相等的点在这个角的平分线上.,
即\( \angle  A \)的平分线经过点\( P\). 
如图\( 1-26\),在\( \triangle ABC \)中,\( AC = BC\),\( \angle  C = 90 ^\circ \),\( AD \)是\( \triangle ABC \)的
角平分线,\( DE \perp AB\),垂足为\( E\). 
\(( 1) \)已知\( CD = 4 cm\),求\( AC \)的长;
\(( 2) \)求证:\( AB = AC + CD\). 
\(( 1) \)解:\( \because AD \)是\( \triangle ABC \)的角平分线,\( DC \perp AC\),\( DE \perp AB\),垂足为\( E\),
\( \therefore DE = CD = 4 cm( \)角平分线上的点到这个角的两边的距离相等.. 
\( \because AC = BC\),
\( \therefore \angle  B = \angle  BAC( \)等边对等角.. 
\( \because \angle  C = 90 ^\circ \),
\( \therefore \angle  B = 1\)
\item \( \times 90 ^\circ = 45 ^\circ \). 
\( \therefore \angle  BDE = 90 ^\circ - 45 ^\circ = 45 ^\circ \). 
\( \therefore BE = DE( \)等角对等边.. 
在等腰直角三角形\( BDE \)中,
\(BD = 2DE2 = 4 2 cm( \)勾股定理.. 
\( \therefore AC = BC = CD + BD =( 4 + 4 2) cm\). 
\(( 2) \)证明:由\(( 1) \)的求解过程易知,
\(Rt \triangle ACD \cong  Rt \triangle AED( HL) \). 
\( \therefore AC = AE( \)全等三角形的对应边相等.. 
\( \because BE = DE = CD\),
\( \therefore AB = AE + BE = AC + CD\). 
随堂练习
已知:如图,在\( R t \triangle A B C \)中,\( \angle  A C B = 9 0 ^\circ \),
\( \angle  B = 60 ^\circ \),\( AD\),\( CE \)是角平分线,\( AD \)与\( CE \)相交于
点\( F\),\( FM \perp AB\),\( FN \perp BC\),垂足分别为\( M\),\( N\). 
求证:\( FE = FD\). 
\(A\)
\(B\)
\(DN C\)
\(F\)
\(E M\)
例\(3\)
\(A\)
\(C B\)
\(D\)
\(E\)
图\( 1-2632\)
数学\(  \)八年级\(  \)下册
习题\( 1.10\)
知识技能
\item 已知:如图,\( \angle  C = 90 ^\circ \),\( \angle  B = 30 ^\circ \),\( AD \)是\( \triangle ABC \)的角平分线. 
求证:\( BD = 2CD\). 
\item 已知:如图,\( \triangle ABC \)的外角\( \angle  CBD \)和\( \angle  BCE \)的平分线相交于点\( F\). 
求证:点\( F \)在\( \angle  DAE \)的平分线上. 
\(A\)
\(B C\)
\(D\)
.第\( 1 \)题\() ( \)第\( 2 \)题.
\(A\)
\(B C\)
\(D E\)
\(F\)
.第\( 3 \)题.
\(A B\)
\(C D\)
\(O\)
\(P\)
\item 已知:如图,\( P \)是\( \angle  AOB \)平分线上的一点,\( PC \perp OA\),\( PD \perp OB\),垂足分别为\( C\),\( D\). 
求证:
\(( 1) OC = OD\);
\(( 2) OP \)是\( CD \)的垂直平分线. 
问题解决
\item 如图,三条公路两两相交,现计划修建一个油库. 
\(( 1) \)如果要求油库到两条公路\( AB\),\( AC \)的距离都相等,那么
如何选择油库的位置?
\(( 2) \)如果要求油库到这三条公路的距离都相等,那么如何选
择油库的位置
\item 请将下面证明中每一步的理由填在括号内:
已知:如图,\( D\),\( E\),\( F \)分别是\( BC\),\( CA\),\( AB \)上的点,\( DE \parallel BA\),\( DF \parallel CA\). 
求证:\( \angle  FDE = \angle  A\). 
证明:\( \because DE \parallel BA( )\),
\( \therefore \angle  FDE = \angle  BFD( ) \). 
\( \because DF \parallel CA( )\),
\( \therefore \angle  BFD = \angle  A( ) \). 
\( \therefore \angle  FDE = \angle  A( ) \). 
回顾与思考
\item 说说作为证明基础的几条基本事实. 
\item 等腰三角形有哪些性质?等边三角形呢?直角三角形呢?它们各自分别有哪些判
定条件?
\item 说说两个直角三角形全等的判定条件,并证明本章中学过的一个判定条件. 
\item 分别说说线段垂直平分线、角平分线的性质定理及其逆定理. 你是怎样发现和证
明它们的?
\item 如何用反证法证明?请举例说明,并与同伴交流. 
\item 请你说出一对互逆命题,并判断它们是真命题还是假命题. 
\item 你认为本章哪些定理的证明方法比较独特?与同伴交流. 
\item 已知底边及底边上的高线,如何用尺规作等腰三角形?已知一直角边和斜边,如
何用尺规作直角三角形?
\item 梳理本章内容,用适当的方式呈现全章知识结构,并与同伴交流. 
\(A\)
\(B C\)
\(D\)
\(E\)
\(F\)
.第\( 1 \)题\()34\)
数学\(  \)八年级\(  \)下册
\item 已知:如图,\( AD \parallel CB\),\( AD = C \)\task 求证:\( \triangle ABC \cong  \triangle CDA\). 
\item 已知:如图,在\( \triangle ABC\)中,\( AB = AC\),点\(D\),\( E\)分别在边\(AC\),\( AB\)上,且\( \angle  ABD = \angle  ACE\),
\(BD \)与\( CE \)相交于点\( O\). 求证:\( ( 1) OB = OC\);\( ( 2) BE = CD\). 
\(A D\)
\(B C\)
.第\( 2 \)题.
\(A\)
\(B C\)
\(E D\)
\(O\)
.第\( 3 \)题.
\(A\)
\(B C\)
\(E D\)
.第\( 4 \)题.
\item 已知:如图,\( BD\),\( CE \)是\( \triangle ABC \)的高,且\( BD = CE\). 求证:\( \triangle ABC \)是等腰三角形. 
\item 在\( \triangle ABC \)中,已知\( \angle  A\),\( \angle  B\),\( \angle  C \)的度数之比是\( 1∶ 2∶ 3\),\( AB = 3\),求\( AC \)的长. 
\item 已知:如图,\( AN \perp OB\),\( BM \perp OA\),垂足分别为\( N\),\( M\),\( OM = ON\),\( BM \)与\( AN \)相交于
点\( P\). 求证:\( PM = PN\). 
\(A\)
\(O M\)
\(N B\)
\(P\)
.第\( 6 \)题\() ( \)第\( 8 \)题.
\(a\)
.第\( 9 \)题.
\(A\)
\(B C\)
\(D\)
\item 已知:\( MN \)是线段\( AB \)的垂直平分线,\( C\),\( D \)是\( MN \)上的两点. 求证:
\(( 1) \triangle ABC\),\( \triangle ABD \)是等腰三角形;\( ( 2) \angle  CAD = \angle  CBD\). 
\item 如图,已知线段\( a\),利用尺规求作以\( a \)为底、以\( 2a \)为高的等腰三角形. 
\item 如图,在\( \triangle ABC \)中,\( \angle  BAC= 90 ^\circ \),\( AB = AC = a\),\( AD \)是\( \triangle ABC \)的高,求\( AD \)的长. 
数学理解
\item 如图,\( \triangle ABC \)的高\( BD \)与\( CE \)相交于点\( O\),\( OD = OE\),\( AO \)的
延长线交\( BC \)于点\( M\),请你从图中找出几对全等的直角三角
形,并给出证明. 
\item 如图,在\( \triangle ABC \)中,\( \angle C = 90 ^\circ \),\( \angle A = 30 ^\circ \),\( AB \)的垂直平分
线分别交\( AB\),\( AC \)于点\( D\),\( E\). 求证:\( AE = 2CE\). 
.第\( 10 \)题.
\(A\)
\(B C\)
\(D\)
\(O M\)
\(E35\)
第一章\(  \)三角形的证明
\(A\)
\(B C\)
\(D\)
\(E\)
.第\( 11 \)题.
\(A B\)
\(C D\)
.第\( 12 \)题\() ( \)第\( 13 \)题.
\(A\)
\(B C\)
\(D\)
\(E\)
\item 如图,在四边形\( BCDE \)中,\( \angle C = \angle BED = 90 ^\circ \),\( \angle B = 60 ^\circ \),延长\( CD\),\( BE\),两线
交于点\task 已知\( CD = 2\),\( DE = 1\),求\( Rt \triangle ABC \)的面积. 
\item 如图,已知\( \angle ACB = \angle BDA = 90 ^\circ \),要使\( \triangle ACB \cong  \triangle BDA\),还需要添加什么条件?
请你选择其中一个加以证明. 
\item 求证:等腰三角形的底角必为锐角. 
\item 如图,在\( \triangle ABC\)中,\( \angle B = 64 ^\circ \),\( \angle BAC = 72 ^\circ \),\( D \)为\( BC \)上
一点,\( DE \)交\( AC \)于点\( F\),且\( AB = AD = DE\),连接\( AE\),
\( \angle E = 55 ^\circ \). 请判断\( \triangle AFD \)的形状,并说明理由. 
联系拓广
\item 如图,在\( \triangle ABC \)中,\( AB = AC\),\( AB \)的垂直平分线交\( AB \)于点\( D\),交\( AC \)于点\( E\). 已
知\( \triangle BCE \)的周长为\( 8\),\( AC - BC = 2\),求\( AB \)与\( BC \)的长. 
.第\( 16 \)题.
\(A\)
\(B C\)
\(D\)
\(E\)
.第\( 17 \)题.
\(A\)
\(B\)
\(E\)
\(F\)
\(C\)
\(D\)
.第\( 18 \)题.
\(c\)
\item 已知:如图,在等边三角形\( ABC \)的三边上分别取点\( D\),\( E\),\( F\),使得\( AD = BE = CF\). 
求证:\( \triangle DEF \)是等边三角形. 
\item 如图,已知线段\( c\),求作等腰直角三角形,使其斜边等于线段\( c( \)保留作图痕迹,不
必写作法.. 
\item 已知等腰三角形底边和腰的长分别为\( 6 \)和\( 5\),求这个等腰三角形的面积. 
\end{liti}
\end{document}
